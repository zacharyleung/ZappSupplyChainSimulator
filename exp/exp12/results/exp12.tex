\documentclass[12pt]{article}

\usepackage{amsmath}
\usepackage{amssymb}
\usepackage{array}
\usepackage{pgfplots}
\usepackage{verbatim}
\usepackage[paperheight=30cm,paperwidth=245mm,margin=3cm]{geometry}

\pgfplotsset{compat=1.12}
\pgfplotsset{width=75mm,height=60mm}

\newcolumntype{M}{>{\centering\arraybackslash}m{\dimexpr.48\linewidth-2\tabcolsep}}

\pagenumbering{gobble} % Remove page numbers (and reset to 1)

\pgfplotscreateplotcyclelist{mystyle}{
line width=0.8pt,black,every mark/.append style={fill=white},mark=otimes*\\%
line width=0.8pt,densely dashed,green!80!black,every mark/.append style={fill=white},mark=square*\\%
line width=0.8pt,densely dotted,blue,every mark/.append style={fill=blue},mark=*\\%
line width=0.8pt,red,mark=star\\%
}

\title{exp12.tex}
\author{Zachary Leung}
\date{2021-06-12}

\begin{document}

\maketitle

\begin{equation*}
\textbf{Notation for $\phi_L$:}
\quad
a_{t}^{h}[\phi_{L}] \triangleq \phi_{L}*a_{t}^{h} + (1-\phi_{L})
\end{equation*}

\begin{center}
\begin{itemize}
\item If $\phi_L$ = 0, then every facility is fully accessible, i.e.,
      $a_t^h[\phi_{L}] = 1$
\item If $\phi_L$ = 1, then every facility has accessibility probability
      equal to the value in the original dataset, i.e.,
      $a_t^h[\phi_{L}] = a_t^h$
\end{itemize}
\end{center}

\subsubsection*{Supply / Demand ratio = 0.8}
\begin{verbatim}
inputFolder = "input/zambia";
\end{verbatim}
\verbatiminput{0.8/0010/out.txt}



\subsubsection*{Supply / Demand ratio = 1.1}
\begin{verbatim}
inputFolder = "input/zambia";
\end{verbatim}
\verbatiminput{1.1/0025/out.txt}

\begin{verbatim}
Note: the relative margin of error (RMOE) is defined as the half-width
of the 95% confidence interval, divided by the mean estimate.
\end{verbatim}




\clearpage
\begin{center}
\begin{tabular}{MM}
\begin{tikzpicture}
  \begin{axis}[
      cycle list name=mystyle,
      title style={align=center},
      title={(a) System Service Level\\(SDR = 0.8)},
      xlabel style={align=center},
      xlabel=Demand Seasonality\\Parameter ($\phi_D$),
      ylabel style={align=center},
      ylabel=~\\System Service Level,
      legend to name={the legend reference},
      legend cell align={left},
      legend style={row sep=3mm},
      legend columns=4,
      xmin=0,xmax=1,ymin=0.7,ymax=1]
      
    % Current
    \addplot
        table[x=DemandParameter,
              col sep=comma,
              y=serv_mean,
              y error=serv_meane]
        {0.8/0010/AmiIpPropPolicyFactoryDelay0.csv};
    \addlegendentry{$4 \times AMI - IP$}

    % Last year
    \addplot
        table[x=DemandParameter,
              col sep=comma,
              y=serv_mean,
              y error=serv_meane]
        {0.8/0010/AmdLastYearIpPropPolicyFactoryDelay0.csv};
    \addlegendentry{$4 \times AMD[-12,-9] - IP$}
    
    % LSI
    \addplot
        table[x=DemandParameter,
              col sep=comma,
              y=serv_mean,
              y error=serv_meane]
        {0.8/0010/LsiAmdIPropPolicyFactoryDelay0.csv};
    \addlegendentry{$4 \times LSI \times AMD - I$}
    
    % Optimization
    \addplot
        table[x=DemandParameter,
              col sep=comma,
              y=serv_mean,
              y error=serv_meane]
        {0.8/0010/OptimizationPolicyDelay0.csv};
    \addlegendentry{$OPT^{16}_{0.99}$}

  \end{axis}
\end{tikzpicture}%
&
\begin{tikzpicture}
  \begin{axis}[
      cycle list name=mystyle,
      title={(b) System Service Level (SDR = 1.1)},
      xlabel=Facility Accessibility Parameter ($\phi_L$),
      ylabel=System Service Level,
      xmin=0,xmax=1,ymin=0.5,ymax=1]

    % Current
    \addplot
        table[x=AccessParameter,
              col sep=comma,
              y=serv_mean,
              y error=serv_meane]
        {1.1/0025/AmiIpPropPolicyFactoryDelay0.csv};

    % Last year
    \addplot
        table[x=AccessParameter,
              col sep=comma,
              y=serv_mean,
              y error=serv_meane]
        {1.1/0025/AmdLastYearIpPropPolicyFactoryDelay0.csv};
    
    % LSI
    \addplot
        table[x=AccessParameter,
              col sep=comma,
              y=serv_mean,
              y error=serv_meane]
        {1.1/0025/LsiAmdIPropPolicyFactoryDelay0.csv};
    
    % Optimization    
    \addplot
        table[x=AccessParameter,
              col sep=comma,
              y=serv_mean,
              y error=serv_meane]
        {1.1/0025/OptimizationPolicyDelay0.csv};

  \end{axis}
\end{tikzpicture}%
\\
~&~
\\[-3mm]
\begin{tikzpicture}
  \begin{axis}[
      cycle list name=mystyle,
      title={(c) Average Facility Inventory (SDR = 0.8)},
      xlabel style={align=center},
      xlabel=Lead Time and Demand Seasonality\\Parameter ($\phi$),
      ylabel={Average Facility Inventory (No.~of Weeks)},
      xmin=0,xmax=1,ymin=0]
      
    % Current
    \addplot
        table[x=PhiParameter,
              col sep=comma,
              y=inv_mean,
              y error=inv_meane]
        {0.8/0010/AmiIpPropPolicyFactoryDelay0.csv};
    \addplot
        table[x=PhiParameter,
              col sep=comma,
              y=inv_mean,
              y error=inv_meane]
        {0.8/0010/AmiIpPropPolicyFactoryDelay1.csv};

    % Last year
    \addplot
        table[x=PhiParameter,
              col sep=comma,
              y=inv_mean,
              y error=inv_meane]
        {0.8/0010/AmdLastYearIpPropPolicyFactoryDelay0.csv};
    \addplot
        table[x=PhiParameter,
              col sep=comma,
              y=inv_mean,
              y error=inv_meane]
        {0.8/0010/AmdLastYearIpPropPolicyFactoryDelay1.csv};
    
    % LSI
    \addplot
        table[x=PhiParameter,
              col sep=comma,
              y=inv_mean,
              y error=inv_meane]
        {0.8/0010/LsiAmdIPropPolicyFactoryDelay0.csv};
    \addplot
        table[x=PhiParameter,
              col sep=comma,
              y=inv_mean,
              y error=inv_meane]
        {0.8/0010/LsiAmdIPropPolicyFactoryDelay1.csv};
    
    % Optimization
    \addplot
        table[x=PhiParameter,
              col sep=comma,
              y=inv_mean,
              y error=inv_meane]
        {0.8/0010/OptimizationPolicyDelay0.csv};
    \addplot
        table[x=PhiParameter,
              col sep=comma,
              y=inv_mean,
              y error=inv_meane]
        {0.8/0010/OptimizationPolicyDelay1.csv};
  \end{axis}
\end{tikzpicture}
%
&
\begin{tikzpicture}
  \begin{axis}[
      cycle list name=mystyle,
      title style={align=center},
      title={(d) Facility Service Level\\Equity (SDR = 1.1)},
      xlabel style={align=center},
      xlabel=Facility Seasonality\\Parameter ($\phi_L$),
      ylabel style={align=center},
      ylabel={Std.~Dev.~of\\Facility Service Level},
      ytick      ={0,0.05,0.10,0.15,0.20,0.25,0.30,0.35,0.40},
      yticklabels={0,0.05,0.10,0.15,0.20,0.25,0.30,0.35,0.40},
      xmin=0,xmax=1,ymin=0,ymax=0.18]
      
    % Current
    \addplot
        table[x=AccessParameter,
              col sep=comma,
              y=stddev_serv_mean,
              y error=stddev_serv_meane]
        {1.1/0025/AmiIpPropPolicyFactoryDelay0.csv};

    % Last year
    \addplot
        table[x=AccessParameter,
              col sep=comma,
              y=stddev_serv_mean,
              y error=stddev_serv_meane]
        {1.1/0025/AmdLastYearIpPropPolicyFactoryDelay0.csv};
    
    % LSI
    \addplot
        table[x=AccessParameter,
              col sep=comma,
              y=stddev_serv_mean,
              y error=stddev_serv_meane]
        {1.1/0025/LsiAmdIPropPolicyFactoryDelay0.csv};
    
    % Optimization    
    \addplot
        table[x=AccessParameter,
              col sep=comma,
              y=stddev_serv_mean,
              y error=stddev_serv_meane]
        {1.1/0025/OptimizationPolicyDelay0.csv};

  \end{axis}
\end{tikzpicture}
%
\end{tabular}
~\\[2mm]
\ref{the legend reference}
\end{center}



\end{document}
