\documentclass[12pt,twocolumn]{article}

\usepackage{hyperref}

\newcommand{\code}[1]{{\scriptsize \texttt{#1}}}

\begin{document}

\tableofcontents

\clearpage

\section{Computer Model of Supply Chain}

SCS was written using an object-oriented programming paradigm.
In an OOP paradigm,
every domain object (entity in the conceptual model of the supply chain)
is represented by a software object in the computer code.

\subsection{Overview}

\subsection{Software Design Principles}

\subsubsection{Object-Oriented Programming}

\subsubsection{Software Design Patterns}

\subsubsection{Random Objects}

\subsection{Controller Objects}

In an OOP paradigm,
every domain object (entity in the conceptual model of the supply chain)
is represented by a software object in the computer code.
Therefore, we define software objects such as
\code{RetailFacility}, \code{Shipment} and \code{Report}
to model entities in the supply chain.
We use \emph{controller objects}
in order to coordinate the interaction
of the domain objects in the supply chain
and to implement the control logic specified in the Conceptual Model chapter.
For example, the controller object
gets a \code{Report} object from the customer facility object,
and arranges for the \code{Report} object
to be received by the supplier facility
after the \code{Report} lead time specified by the
\code{ReplenishmentPolicy} object.

\subsection{Supply Chain Topology}

The \code{Facility} class is a superclass of the
\code{NationalFacility}, \code{RegionalFacility}
and \code{RetailFacility} classes.

The \code{Topology} class stores a mapping of each \code{RetailFacility}
to its \code{RegionalFacility}.

\subsection{Exogenous Supply and Demand}

National supply schedule

Demand model

\subsection{Inventory Replenishment Process}

\subsubsection{Basic Process}

\subsubsection{Inventory Replenishment Constraints}

Lead time model

\subsubsection{Intermediate Stocking Configuration}

\subsubsection{Cross-Docking Configuration}

\subsubsection{Reports and Shipments}




\section{Conceptual Model of Zambia Supply Chain}

Here is a description of the conceptual model of the Zambia supply chain





\section{Computer Model of Zambia Supply Chain}

We use SCS to build a computer model of the Zambia supply chain.

Here is how we map the entities
in the conceptual model of the Zambia supply chain
to entities in the conceptual model of the SCS supply chain.
Show a table.

\subsection{Input Files}

Here are the input files for the supply chain topology,
the demand mean, the demand forecast accuracy,
the regional facility shipment schedule,
primary and secondary lead times.

\subsection{Replenishment Policies}

We implement the order-up-to intermediate stocking policy in this class,
the order-up-to cross-docking policy in this class,
the clairvoyant cross-docking policy in this class
and the optimization-based cross-docking in this class.




\end{document}
