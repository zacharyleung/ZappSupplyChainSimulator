\chapter[Conceptual Model (\scs)]{Conceptual Model of \scs\ Supply Chain}
\label{chapter:conceptual-model}


\section{Overview}
\label{section:conceptual-model->overview}

In this chapter,
we define the structure of the supply chain
that is implemented in computer code by \scs.

The \scs\ supply chain is a conceptual model of
a country's supply chain for a single product.
In \autoref{section:conceptual-model->discrete-time-framework}
we define the discrete time framework that is used by \scs\
to model the evolution of the state of the supply chain.
We define the supply chain topology
in \autoref{section:conceptual-model->supply-chain-topology},
the exogenous supply and demand for inventory
in \autoref{section:conceptual-model->supply-and-demand},
the inventory replenishment process
in \autoref{section:conceptual-model->inventory-replenishment},
and the sequence of events in each period in
\autoref{section:conceptual-model->sequence-of-events}.

\paragraph{Notation}
In this chapter,
we use small capitals, e.g.\ \entity{national facility},
to denote a conceptual entity that is represented in the supply chain.





\section{Discrete Time Framework}
\label{section:conceptual-model->discrete-time-framework}

% Based on
% http://en.wikipedia.org/wiki/Discrete_event_simulation
\scs\ is a \emph{discrete-event simulation} of a supply chain.
Wikipedia defines a discrete-event simulation as follows:
\begin{quotation}
A discrete-event simulation models the operation of a system
as a discrete sequence of events in time.
Each event occurs at a particular instant in time
and marks a change of state in the system.
Between consecutive events, no change in the system is assumed to occur.
\end{quotation}

% Based on
% http://en.wikipedia.org/wiki/Discrete_time_and_continuous_time
\scs\ is also a \emph{discrete time} simulation of a supply chain.
That is, conceptually we divide time into \emph{periods},
with a \emph{period} defined as the smallest unit of time
that is represented in the simulation.
For example, if we decide to model a year using 48 periods,
then one period corresponds to a time interval of 1/48-th of a year;
whereas if we choose to model a year using 365 periods,
then one period corresponds to a time interval of one day.
It is assumed that changes in the state of the supply chain
occur at points in time that correspond to the beginning/end of a period.





\section{Supply Chain Topology}
\label{section:conceptual-model->supply-chain-topology}

The supply chain topology can be represented by a graph,
in which each facility corresponds to a node in the graph,
and each (supplier, customer) relationship corresponds to
a directed edge in the graph.

The \scs\ supply chain is a three-tier supply chain,
consisting of exactly one \entity{national facility},
one or more \entity{regional facilities},
and one or more \entity{retail facilities}.
It is assumed that each \entity{retail facility}
is located in a unique geographical region,
which is served by a corresponding \entity{regional facility}.

It is assumed that direct inventory shipments only occur
from facilities in one tier of the supply chain
to facilities in the tier directly below.
In other words,
the \entity{national facility}
is the supplier of each \entity{regional facility};
and each \entity{retail facility}
is supplied by its \entity{regional facility}.

The structure of the supply chain topology is illustrated in 
\autoref{figure:conceptual-model->supply-chain-topology}.

\begin{figure}[h!]
\centering
\begin{tikzpicture}
    [exoge/.style={rectangle,minimum width=20mm,align=center,
                   text width=20mm,font=\footnotesize},
     node distance=10mm]
  % facilities
  \node[exoge] (sup)                               {Exogeneous Supply};
  \node[facility] (nat)  [below=of sup]               {National Facility};
  \node[facility] (reg1) [below=of nat,xshift=-30mm]  {Regional Facility};
  \node[facility] (reg2) [below=of nat,xshift=30mm]   {Regional Facility};
  \node[facility] (ret1) [below=of reg1,xshift=-15mm] {Retail Facility};
  \node[facility] (ret2) [below=of reg1,xshift=15mm]  {Retail Facility};
  \node[facility] (ret3) [below=of reg2,xshift=-15mm] {Retail Facility};
  \node[facility] (ret4) [below=of reg2,xshift=15mm]  {Retail Facility};
  \node[exoge] (dem1) [below=of ret1]              {Exogeneous Demand};
  \node[exoge] (dem2) [below=of ret2]              {Exogeneous Demand};
  \node[exoge] (dem3) [below=of ret3]              {Exogeneous Demand};
  \node[exoge] (dem4) [below=of ret4]              {Exogeneous Demand};

  % inventory flows
  \draw[medarrow] (sup) -- (nat);
  \draw[medarrow] (nat) -- (reg1);
  \draw[medarrow] (nat) -- (reg2);
  \draw[medarrow] (reg1) -- (ret1);
  \draw[medarrow] (reg1) -- (ret2);
  \draw[medarrow] (reg2) -- (ret3);
  \draw[medarrow] (reg2) -- (ret4);
  \draw[medarrow] (ret1) -- (dem1);
  \draw[medarrow] (ret2) -- (dem2);
  \draw[medarrow] (ret3) -- (dem3);
  \draw[medarrow] (ret4) -- (dem4);
\end{tikzpicture}
\caption{Example to illustrate the supply chain topology of \scs.
Arrows denote the direction of inventory flows.}
\label{figure:conceptual-model->supply-chain-topology}
\end{figure}




\section{Exogenous Supply and Demand}
\label{section:conceptual-model->supply-and-demand}

The exogenous supply arrives at the \entity{national facility},
while the exogenous demand arrives at each \entity{retail facility}.
This is illustrated in
\autoref{figure:conceptual-model->supply-chain-topology}.

The \entity{national supply schedule}
specifies both the timing and the quantity
of inventory arriving at the \entity{national facility}.
It is assumed that the supply of inventory
received by the \entity{national facility} is exogenous.

At each \entity{retail facility},
inventory is depleted by the demand
induced by customers who arrive at the facility.
In general, the demand is random,
and generated from a probability distribution
which is specified by the \entity{demand model}.
The demand is assumed to be exogenous,
i.e.\ it is statistically independent
of the state of the supply chain
and of the inventory replenishment policy used.

In the supply chain management literature,
there are two different types of assumptions made about unmet demand:
either unmet demand is backordered
(i.e.\ customers remain waiting for the product
and the demand is satisfied when inventory arrives in a future period),
or unment demand is lost forever.
\scs\ assumes that unmet demand is lost forever.

To define the \entity{demand model} more precisely,
let us denote by $\rvd_{rt}$
the random quantity demanded at retail facility $r$ during period $t$.
The \entity{demand model} specifies the joint probability distribution
of the demand for each retail facility and each period,
i.e.\ the set of random variables $\{\rvd_{rt}\}_{r,t}$
over all retail facilities $r$ and all periods $t$.

The \entity{demand model} is also able to provide
a demand forecast of future demand $\rvd_{ru}$
given all information revealed by the current time period $t$,
which we denote by $\info_t$.
Certain \entity{inventory replenishment policies}
may use compute a replenishment quantity
that depends on the demand forecasts $\rvd_{ru}$.
For example,
an order-up-to policy
that orders up to the mean demand in the next 12 periods
would ship a replenishment quantity given by
\begin{align*}
    q
  &=
    \text{expected demand in } [t, t+12) - \text{inventory position}_t
\\
  &=
    \sum_{u=t}^{t+11} \cEX{\rvd_{ru}}{\info_t} - ip_t.
\end{align*}






\section{Inventory Replenishment Process}
\label{section:conceptual-model->inventory-replenishment} 

We define the basic inventory replenishment process
in \autoref{conceptual-model->inventory-replenishment:generic-(supplier-customer)}.
We introduce inventory replenishment constraints
in \autoref{section:conceptual-model->inventory-replenishment-constraints}.
The \scs\ supply chain can be operated
in either an intermediate stocking configuration
or a cross-docking configuration.
We define the inventory replenishment process
for an intermediate stocking configuration
in \autoref{subsection:conceptual-model->inventory-replenishment->istock},
while we define the process
for a cross-docking configuration
in \autoref{subsection:conceptual-model->inventory-replenishment->xdock}.






\subsection{Basic Inventory Replenishment Process}
\label{conceptual-model->inventory-replenishment:generic-(supplier-customer)}

In this section,
we define the basic inventory replenishment process,
i.e.\ the process by which a customer facility
receives inventory replenishment from a dedicated supplier facility.
We refer to this arrangement as a (supplier, customer) relationship.

In order to define the inventory replenishment process,
we first need to define the following entities:
\begin{description}
  \item[Report]
      The \entity{report} submitted by the customer facility
      is a record of information that is relevant
      in calculating a replenishment shipment quantity, including:
      \begin{itemize}
        \item The customer facility's current inventory level
        \item Shipments in the inventory pipeline to the customer facility
        \item Past demand and consumption at the customer facility
      \end{itemize}
  \item[Replenishment policy]
      When a supplier facility has received a \entity{report}
      from a customer facility,
      the supplier facility uses the \entity{replenishment policy}
      to calculate the shipment quantity
      based on a subset of the following inputs:
      \begin{itemize}
        \item the received \entity{report}
        \item the forecast for future \entity{demand} at the customer
        \item the \entity{lead time} forecast for past, current and future
            shipments to the customer
      \end{itemize}
      A \entity{replenishment policy} need not take into account
      all of the above factors to compute a shipment quantity.
      For example, an order-up-to policy
      only depends on the customer facility \entity{report}.
  \item[Shipment]
      A shipment is some number of units of the product
      that is packaged by the supplier facility into a common lot
      and shipped to the customer facility.
\end{description}

The basic inventory replenishment process is the following:
\begin{enumerate}
  \item The customer facility sends a \entity{report}
      to its supplier facility.
  \item The supplier facility processes the \entity{report}
      according to the \entity{inventory replenishment policy}
      and computes a replenishment quantity,
      and makes an \entity{shipment} to the customer facility.
  \item The \entity{shipment} arrives at the customer facility.
\end{enumerate}
This is illustrated in
\autoref{figure:inventory-replenishment-process}.


\begin{figure}[h!]
\centering
\begin{tikzpicture}[node distance=30mm]
  \node[facility] (sup)                 {Supplier Facility};
  \node[facility] (cust) [right=of sup] {Customer Facility};

  \draw[medarrow,dashed]
      (cust.north west) -- 
      node[auto,mail,above] {Report}
      (sup.north east);

  \draw[medarrow]
      (sup.south east) -- 
      node[auto,mail,below] {Shipment}
      (cust.south west);
\end{tikzpicture}
\\[5mm]
\begin{tikzpicture}[
    xscale=1.3,
    node distance=15mm,
    ]
  % space between event labels and interval arrow
  \def \myyshift{2mm}

  \foreach \i in {-3,...,4} {
    \draw (\i cm,3pt)
        node (uptic\i) {}
        -- (\i cm,-3pt)
        node (downtic\i) {};
  }
  \draw[->,line width=1pt] (-3.2,0) -- (4.8,0);

  \node[timelabel]
       (reportSentTime) at (downtic-2) {$t - r$};
  \node[event]    (reportSentLabel) [above=of uptic-2]
      {Customer Facility sends report};
  \draw[medarrow] (reportSentLabel) -- (uptic-2);

  \node[timelabel]
      (reportSentTime) at (downtic0) {$t$};
  \node[event]     (reportReceivedLabel) [above=of uptic0]
      {Supplier Facility receives report};
  \draw[medarrow] (reportReceivedLabel) -- (uptic0);
  \node[event] (shipmentSentLabel) [below=of reportSentTime]
      {Supplier Facility sends shipment};
  \draw[medarrow] (shipmentSentLabel) -- (reportSentTime);

  \node[timelabel]
      (shipmentSentTime) at (downtic4) {$t + \ell_t$};
  \node[event] (shipmentReceivedLabel) [below=of shipmentSentTime]
      {Customer Facility receives shipment};
  \draw[medarrow] (shipmentReceivedLabel) -- (shipmentSentTime);

  \draw[interval]
      ([yshift=\myyshift]reportSentLabel.north) --
      node[auto,above] {Report lead time}
      ([yshift=\myyshift]reportReceivedLabel.north);

  \draw[interval]
      ([yshift=-\myyshift]shipmentSentLabel.south) --
      node[auto,below] {Shipment lead time}
      ([yshift=-\myyshift]shipmentReceivedLabel.south);
\end{tikzpicture}
\caption{Illustration of the inventory replenishment process.}
\label{figure:inventory-replenishment-process}
\end{figure}

In order to define the schedule of events
in the inventory replenishment process,
we first need to define the following concepts:
\begin{description}
  \item[Shipment schedule] 
      The \entity{shipment schedule} indicates which are the periods
      in which the supplier facility can make a \entity{shipment}
      to the customer facility.
  \item[Shipment lead time]
      The \entity{shipment} lead time is the number of periods of delay
      between the supplier facility sending a \entity{shipment}
      and the customer facility receiving the \entity{shipment}.
      The \entity{shipment} lead time may be random
      and depend on the period in which the \entity{shipment} is sent.
      We denote the \entity{shipment} lead time at period $t$
      by $\ell_t$ periods.
  \item[Report lead time]
      The \entity{report} lead time is the number of periods of delay
      between the customer facility sending a \entity{report}
      and the customer facility receiving the \entity{report}.
      The \entity{report} lead time is assumed to be constant,
      and we denote the \entity{report} lead time by $r$ periods.
\end{description}

Suppose that the \entity{shipment schedule}
indicates that the supplier facility
is eligible to make a \entity{shipment}
to the customer facility at the beginning of period $t$.
The schedule of events in the inventory replenishment process
is defined as follows:
\begin{description}
  \item[Period $t - r$:]
      The customer facility submits a \entity{report}
      to the supplier facility.
  \item[Period $t$:]
      The supplier facility receives the \entity{report}
      from the customer facility,
      and makes a \entity{shipment} to the customer facility.
  \item[Period $t + \ell_t$:]
      The customer facility receives the \entity{shipment}
      from the supplier facility.
\end{description}
This is illustrated in 
\autoref{figure:inventory-replenishment-process}.






\subsection{Inventory Replenishment Constraints}
\label{section:conceptual-model->inventory-replenishment-constraints}

For the sake of clarity,
we will refer to shipments
from the \entity{national facility} to \entity{regional facilities}
as \emph{primary distribution};
and refer to shipments
from \entity{regional facilities} to \entity{retail facilities}
as \emph{secondary distribution}.
This is illustrated in
\autoref{figure:conceptual-model->inventory-replenishment-constraints->primary-secondary-distribution}.

\begin{figure}[h!]
  \centering
  \def \myyshift {0mm}
  \tikzset{node distance=25mm}
  \centering
  \begin{tikzpicture}
    % Facilities
    \node[facility] (nat)                {National Facility};
    \node[facility] (reg) [right=of nat] {Regional Facility};
    \node[facility] (ret) [right=of reg] {Retail Facility};

    % Inventory flows
    \draw[invflow]
        (nat) -- 
        node[auto,above,text width=20mm,align=center,font=\footnotesize]
            {Primary distribution}
        (reg);
    \draw[invflow]
        (reg) -- 
        node[auto,above,text width=20mm,align=center,font=\footnotesize]
            {Secondary distribution}
        (ret);
  \end{tikzpicture}
  \caption{Illustration of primary and secondary distribution.
    Arrows denote the direction of inventory flows.
  }
  \label{figure:conceptual-model->inventory-replenishment-constraints->primary-secondary-distribution}
\end{figure}



In the same way,
we can define the primary and secondary \entity{report} lead times,
and the primary and secondary \entity{shipment} lead times,
and primary and secondary \entity{shipment schedules}.
This is illustrated in
\autoref{figure:conceptual-model->inventory-replenishment:primary-secondary-lead-times}.

\begin{figure}[h!]
  \centering
  \def \myyshift {0mm}
  \tikzset{node distance=25mm}
  \centering
  \begin{tikzpicture}
    % Facilities
    \node[facility] (nat)                {National Facility};
    \node[facility] (reg) [right=of nat] {Regional Facility};
    \node[facility] (ret) [right=of reg] {Retail Facility};

    % Report flows
    \draw[reportflow]
        (reg.north west) -- 
        node[auto,above,text width=20mm,align=center,font=\footnotesize]
            {Primary report lead time}
        (nat.north east);

    \draw[reportflow]
        (ret.north west) -- 
        node[auto,above,text width=20mm,align=center,font=\footnotesize]
            {Secondary report lead time}
        (reg.north east);
    % Inventory flows
    \draw[invflow]
        (nat.south east) -- 
        node[auto,below,text width=20mm,align=center,font=\footnotesize]
            {Primary shipment lead time}
        (reg.south west);
    \draw[invflow]
        (reg.south east) -- 
        node[auto,below,text width=20mm,align=center,font=\footnotesize]
            {Secondary shipment lead time}
        (ret.south west);
  \end{tikzpicture}
  \caption{Illustration of primary/secondary report/shipment lead times.}
  \label{figure:conceptual-model->inventory-replenishment:primary-secondary-lead-times}
\end{figure}




It is assumed that the primary and secondary \entity{shipment schedules}
are invariable.

It is assumed that the \entity{report} lead time is constant and deterministic,
and that the primary \entity{report} lead time
is equal to the secondary \entity{report} lead time.

In general, the \entity{shipment} lead times are random
and can be non-stationary.





\subsection{Intermediate Stocking Configuration}
\label{subsection:conceptual-model->inventory-replenishment->istock}

In an \entity{intermediate stocking configuration},
each \entity{regional facility} maintains a common pool of inventory,
from which replenishment shipments are made
to all \entity{retail facilities} in its region.
This is illustrated in
\autoref{figure:conceptual-model->istock-vs-xdock}.

\begin{figure}[h!]
\centering
\def \myyshift {3mm}
\tikzset{node distance=25mm}
\begin{subfigure}{\textwidth}
  \centering
  \begin{tikzpicture}
    % Facilities
    \node[facility] (nat)                {National Facility};
    \node[facility] (reg) [right=of nat] {Regional Facility};
    \node[facility] (ret) [right=of reg] {Retail Facility};

    % Reports
    \draw[->,line width=2pt,dashed]
        ([yshift=\myyshift]ret.west) --
        node[auto,mail,above] {Report}
        ([yshift=\myyshift]reg.east);
    \draw[->,line width=2pt,dashed]
        ([yshift=\myyshift]reg.west) --
        node[auto,mail,above] {Report}
        ([yshift=\myyshift]nat.east);

    % Shipments
    \draw[->,line width=2pt]
        ([yshift=-\myyshift]nat.east) --
        node[auto,mail,below] {Shipment}
        ([yshift=-\myyshift]reg.west);
    \draw[->,line width=2pt]
        ([yshift=-\myyshift]reg.east) --
        node[auto,mail,below] {Shipment}
        ([yshift=-\myyshift]ret.west);
  \end{tikzpicture}
  \captionof{figure}{Intermediate stocking configuration}
\end{subfigure}
\\
\begin{subfigure}{\textwidth}
  \centering
  \begin{tikzpicture}
    \node[facility] (nat)                {National Facility};
    \node[facility,fill opacity=0.2] (reg) [right=of nat] {Regional Facility};
    \node[facility] (ret) [right=of reg] {Retail Facility};

    % Report
    \draw[->,line width=2pt,dashed]
        ([yshift=\myyshift]ret.west) --
        node[auto,mail,above] {Report}
        ([yshift=\myyshift]nat.east);

    % Shipment
    \draw[->,line width=2pt]
        ([yshift=-\myyshift]nat.east) --
        node[auto,mail,below] {Shipment}
        ([yshift=-\myyshift]ret.west);
  \end{tikzpicture}
  \captionof{figure}{Cross-docking configuration}
\end{subfigure}
\caption{
  Illustration of the flow of information and inventory
  in the intermediate stocking and cross-docking
  supply chain configurations.
}
\label{figure:conceptual-model->istock-vs-xdock}
\end{figure}



In an intermediate stocking configuration,
there are two types of (supplier, customer) relationships.
\begin{itemize}
\item
In the (\entity{national facility}, \entity{regional facility}) relationship,
the \entity{report} lead time is the primary \entity{report} lead time,
the \entity{shipment} lead time is the primary \entity{report} lead time,
and the \entity{shipment schedule} is the primary \entity{shipment schedule}.
\item
In the (\entity{regional facility}, \entity{retail facility}) relationship,
the \entity{report} lead time is the secondary \entity{report} lead time,
the \entity{shipment} lead time is the secondary \entity{report} lead time,
and the \entity{shipment schedule} is the secondary \entity{shipment schedule}.
\end{itemize}





\subsection{Cross-docking Replenishment}
\label{subsection:conceptual-model->inventory-replenishment->xdock}

In a cross-docking configuration,
a \entity{retail facility} submits a \entity{report}
which passes through its respective \entity{regional facility}
to the \entity{national facility};
upon receipt of the \entity{report},
the \entity{national facility} makes a \entity{shipment}
that passes through the \entity{regional facility}
to be received by the \entity{retail facility}.
When the supply chain is operated in a cross-docking configuration,
Any inventory held in a \entity{regional facility} is consigned
for a particular \entity{retail facility}.
This is illustrated in
\autoref{figure:conceptual-model->istock-vs-xdock}.

In a cross-docking configuration,
there is only onetype of (supplier, customer) relationship
which is (\entity{national facility}, \entity{retail facility}).
It is assumed that once a \entity{regional facility}
receives a \entity{shipment} intended for
a \entity{retail facility} in its region,
the \entity{regional facility} immediately initiates delivery,
so the \entity{shipment} is received at the \entity{retail facility}
after the secondary \entity{shipment} lead time.
In this relationship,
the \entity{report} lead time is
the sum of the primary and secondary \entity{report} lead times;
the \entity{shipment} lead time is
the sum of the primary and secondary \entity{shipment} lead times;
and the \entity{shipment schedule} is
the primary \entity{shipment schedule}.





\section{Sequence of Events in Each Period}
\label{section:conceptual-model->sequence-of-events}

The supply chain is operated
in either an \entity{intermediate stocking} configuration
or a \entity{cross-docking configuration}.
In either configuration,
the sequence of events in each period follows the following pattern:
\begin{enumerate}
\item
\textbf{Customer facilities submit reports.}
For each (supplier, customer) relationship,
if the current period is a reporting period,
then a customer facility submits a \entity{report} to a supplier facility.
The order of \entity{report} submission
is from facilities in the lowest tier to the highest tier,
i.e.\ \entity{retail facilities} submit \entity{reports}
before \entity{regional facilities}.
\item
\textbf{Facilities receive shipments and make shipments.}
For each facility,
receive \entity{shipments} and make \entity{shipments} (if applicable).
First, the exogeneous supplier makes a \entity{shipment}
to the \entity{national facility}
according to the \entity{national replenishment policy}.
Then in the order of highest tier to the lowest tier,
facilities receive \entity{shipments}
and make \entity{shipments} (if applicable).
In other words,
first the \entity{national facility} receives and makes \entity{shipments},
then the \entity{regional facilities} receive and make \entity{shipments},
and finally the \entity{retail facilities} receive \entity{shipments}.
\item
\textbf{Customers arrive at retail facilities.}
Customers arrive at \entity{retail facilities},
depleting inventory and incurring unmet demand (if applicable).
\end{enumerate}



The pseudocode for the detailed sequence of events
in an \entity{intermediate stocking} configuration
is shown in \autoref{algo:conceptual-model->sequence->istock};
while the pseudocode for the detailed sequence of events
in an \entity{cross-docking} configuration
is shown in \autoref{algo:conceptual-model->sequence->xdock}.


\begin{algorithm}[h!]
\begin{algorithmic}
\State $t \gets$ start period
\While{$t \leq$ end period}
    \State \algocomment{submit reports}
    \ForAll{(regional, retail) is a SCR}
        \If{is reporting period for customer facility}
            \State customer facility submits report to supplier facility
        \EndIf
    \EndFor
    \ForAll{(national, regional) is a SCR}
        \If{is reporting period for customer facility}
            \State customer facility submits report to supplier facility
        \EndIf
    \EndFor
    \State \algocomment{receive shipments and make shipments}
    \State exogeneous supplier
        makes {shipments} to {national facility}
    \State {national facility} receives {shipments}
    \ForAll{regional facility}
        \State {regional facility} receives {shipments}
        \State {regional facility} makes {shipments}
            to {retail facilities}
    \EndFor
    \ForAll{retail facility}
        \State {retail facility} receives {shipments}
    \EndFor
    \State \algocomment{demand arrives}
    \ForAll{retail facility}
        \State demand arrives at {retail facility}
    \EndFor
\EndWhile
\end{algorithmic}
\caption{Pseudocode for sequence of events in each period
for the intermediate stocking simulator.}
\label{algo:conceptual-model->sequence->istock}
\end{algorithm}





\begin{algorithm}[h!]
\begin{algorithmic}
\State $t \gets$ start period
\While{$t \leq$ end period}
    \State \algocomment{submit reports}
    \ForAll{retail facility}
        \If{is reporting period for customer facility}
            \State retail facility submits report to national facility
        \EndIf
    \EndFor
    \State \algocomment{receive shipments and make shipments}
    \State exogeneous supplier
        makes {shipments} to {national facility}
    \State {national facility} receives {shipments}
    \State {national facility} makes {shipments}
            to {retail facilities}
    \ForAll{retail facility}
        \State {retail facility} receives {shipments}
    \EndFor
    \State \algocomment{demand arrives}
    \ForAll{retail facility}
        \State demand arrives at {retail facility}
    \EndFor
\EndWhile
\end{algorithmic}
\caption{Pseudocode for sequence of events in each period
for the cross-docking simulator.}
\label{algo:conceptual-model->sequence->xdock}
\end{algorithm}

\clearpage





\section{Simulation Replication}
\label{section:conceptual-model->simulator-replication}

A simulation run of \scs\ can be characterized by three separate components:
\begin{itemize}
  \item Exogeneous components
  \item Replenishment process components
  \item Initial state
  \item Simulation run parameters
\end{itemize}

The following components of the supply chain
are assumed to be exogeneous,
and independent of the replenishment process
and the simulation run parameters:
\begin{itemize}
  \item supply chain topology
  \item shipment lead time model
  \item report lead time model
  \item demand model model
  \item replenishment policy for national facility
  \item national facility to regional facility shipment schedule
\end{itemize}

The replenishment process depends on the supply chain configuration.
\begin{itemize}
  \item
    In an intermediate stocking configuration,
    the replenishment process is defined by:
    \begin{itemize}
      \item replenishment policy for regional facilities
      \item replenishment policy for retail facilities
      \item regional facility to retail facility shipment schedule
    \end{itemize}
  \item
    In a cross-docking configuration,
    the replenishment process depends only on the retail replenishment policy.
\end{itemize}

The initial state of the supply chain is defined by
\begin{itemize}
  \item The inventory level at the \entity{national facility}
  \item The inventory level at each \entity{regional facility}
    (only in the intermediate stocking configuration)
  \item The inventory level at each \entity{retail facility}
\end{itemize}

The simulation run parameters are:
\begin{itemize}
  \item initial inventory level at national, regional, retail facilities
  \item simulation start and end periods
  \item random seed (i.e.\ the number used to initialize
    a pseudorandom number generator)
\end{itemize}




