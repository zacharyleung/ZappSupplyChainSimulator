\chapter[Computer Model (\scs)]{Computer Model of Supply Chain}
\label{chapter:implementation}

\section{Overview}

In this chapter,
we describe the computer model of the supply chain.
In other words,
we describe the software implementation
of the conceptual model of the supply chain
which was described in \autoref{chapter:conceptual-model}.


In \autoref{section:scs->computer-model->java-programming},
we give the reader a brief introduction
to the Java\texttrademark\  programming language.
In \autoref{section:scs->computer-model->software-engineering-tools}
we explain some of the software engineering tools
that we used in the software implementation and documentation of \scs.
In \autoref{section:scs->computer-model->controller-objects},
we describe the controller objects which implement the supply chain logic
which was described in \autoref{chapter:conceptual-model}.
The following three sections each describe
the computer implementation of an element in the supply chain,
the supply chain topology in
\autoref{section:scs->computer-model->supply-chain-topology},
the exogeneous supply and demand in 
\autoref{section:scs->computer-model->supply-and-demand},
the inventory replenishment process in
\autoref{section:scs->computer-model->replenishment-process}.
While reading these three sections,
it may be useful for the reader to refer to the
corresponding section in \autoref{chapter:conceptual-model}.
Finally, in \autoref{section:scs->computer-model->design-of-experiments}
we describe how to design computer experiments.

\paragraph{Notation}
In this chapter,
we use typewriter font, e.g.\ \code{NationalFacility},
to denote the name of a Java class.
The code lisitings, e.g.\ \autoref{figure:hello-world}
also use the typewriter font.





\section{Essentials of Java Programming}
\label{section:scs->computer-model->java-programming}

\subsection{Hello World}

In the Java programming language,
every application must contain a \code{main} method.
This is the entry point of the application
and will subsequently invoke all the other methods required by your program.
For example, the famous ``Hello World'' program in Java
is shown in \autoref{figure:hello-world}.
For more information about the \code{main} method,
see \citet{java-tutorials->hello-world}.



\begin{figure}[h!]
\begin{lstlisting}
public class HelloWorld {
    public static void main(String[] args) {
        System.out.println("Hello world");
    }
}
\end{lstlisting}
\caption{The famous ``Hello world'' program in Java}
\label{figure:hello-world}
\end{figure}





\subsection{Object-Oriented Programming}

In an object-oriented programming paradigm,
the computer program creates software objects,
which are models of real-world objects.
A software object has an internal state
and methods which model
the interaction of the real-world object
with other objects or the external environment.
The computer program accomplishes its goal
by guiding the interaction of its software objects.

In the Java programming language,
a software object is created by instantiating a class.
A \emph{class} is a blueprint or prototype that defines
how to create a software object,
and what is the state and behavior of the software object.
For example, the code in
\autoref{figure:scs->computer-model->dog-class}
is an example of a class definition for the class \code{Dog}.
The class definition
defines the state of a software dog object (the dog's breed and name)
and its behavior (the dog can print its breed and name to the console).

\begin{figure}[h!]
\begin{lstlisting}
public class Dog {
    private String breed;
    private String name;

    public Dog(String breed, String name) {
        this.breed = breed;
        this.name = name;
    }

    public void print() {
        System.out.printf("Dog{breed=%s,name=%s}%n", breed, name);
    }
}
\end{lstlisting}
\caption{Definition of the class \protect\code{Dog}.}
\label{figure:scs->computer-model->dog-class}
\end{figure}

For more information about OOP concepts,
we refer the reader to \citet{java-tutorials->oop-concepts}.


\scs\ is written using an OOP paradigm.
In other words, each entity in the conceptual model of the supply chain
is modeled by a software object in the simulation
(see \autoref{table:scs->computer-model->supply-chain-topology}).

\begin{table}[h!]
\centering
\small
\begin{tabular}{ll}
\toprule
  Java class & Conceptual entity \\
\midrule
    \code{Topology}
  &
    Supply chain topology
\\
    \code{NationalFacility}
  &
    National facility
\\
    \code{RegionalFacility}
  &
    Regional facility
\\
    \code{RetailFacility}
  &
    Retail facility
\\
    \code{Shipment}
  &
    Shipment
\\
    \code{Report}
  &
    Report
\\
    \code{NationalSupplySchedule}
  &
    Supply schedule for national facility
\\
    \code{DemandModel}
  &
    Demand model
\\
    \code{LeadTime}
  &
    Lead time model
\\
    \code{AbstractReplenishmentPolicy}
  &
    Inventory replenishment policy
\\
\bottomrule
\end{tabular}
\caption{Java classes that implement conceptual entities
in the supply chain topology.}
\label{table:scs->computer-model->supply-chain-topology}
\end{table}




Java supports an object-oriented programming concept
which is refered to as \emph{inheritance}.
This is explained by the Oracle lesson on inheritance
\cite{java-tutorials->inheritance}
as follows:
\begin{quotation}
Certain objects can have common states and behaviors.
For example, mountain bikes, road bikes, and tandem bikes
share certain characteristics of bicycles such as
current speed and current gear.
Yet each also defines additional features that make them different.
Object-oriented programming allows classes to \emph{inherit}
commonly used state and behavior from other classes.
In this example,
\code{Bicycle} now becomes the superclass
of \code{MountainBike}, \code{RoadBike}, and \code{TandemBike}.
\end{quotation}




\subsection{Packages}


A large software project can have a large number of Java classes.
For example, the implementation of \scs\ has more than 80 classes.
If the classes were not organized in some way,
it would be hard to for a programmer
to know what classes are related
and how classes work together to achieve a desired functionality.

In order to manage large and complex software projects,
people use a software design technique known as \emph{modular programming}.
With modular programming, the software project
is divided into distinct modules,
such that each module is responsible for
one and only one aspect of the desired functionality.

Java supports modular programming through its package mechanism
\citep{java-tutorials->packages}.
The computer implementation of \scs\ is composed of the packages
shown in \autoref{table:scs->computer-model->packages}.

\begin{table}[h!]
\centering
\small
\begin{tabular}{ll}
\toprule
  Java package & Concern \\
\midrule
    \code{com.gly.random}
  & Random variables
\\
    \code{com.gly.util}
  & Utility classes
\\
    \code{com.gly.scs.data}
  & The repository classes
\\
    \code{com.gly.scs.demand}
  & Demand models
\\
    \code{com.gly.scs.domain}
  & Entities in the supply chain
\\
    \code{com.gly.scs.leadtime}
  & Lead time models
\\
    \code{com.gly.scs.main}
  & Classes with \code{main} methods
\\
    \code{com.gly.replen}
  & Replenishment policies
\\
    \code{com.gly.scs.sched}
  & Shipment schedules
\\
    \code{com.gly.sim}
  & Controller objects
\\
\bottomrule
\end{tabular}
\caption{Packages in the \scs\ computer code.}
\label{table:scs->computer-model->packages}
\end{table}





\section{Software Engineering Tools}
\label{section:scs->computer-model->software-engineering-tools}

\subsection{Unified Modeling Language}

In this chapter,
we use UML class diagrams to visualize the fields and methods of classes
and the interrelationships between classes.
UML class diagrams are a standard tool in the field of software engineering.
Wikipedia \citep{wikipedia->uml}
describes the Unified Modeling Language (UML) as follows:
\begin{quote}
The Unified Modeling Language (UML) is a general-purpose modeling language in the field of software engineering, which is designed to provide a standard way to visualize the design of a system.
\end{quote}

In particular, one of the components of UML is the UML class diagram
which Wikipedia \citep{wikipedia->uml-class-diagram} desribes as follows:
\begin{quote}
In software engineering, a class diagram in the Unified Modeling Language (UML) is a type of static structure diagram that describes the structure of a system by showing the system's classes, their attributes, operations (or methods), and the relationships among objects.
\end{quote}

A simple example of the UML class diagram of the \code{Dog} class
defined above in 
\autoref{figure:scs->computer-model->dog-class}
is shown in
\autoref{figure:scs-computer-model->dog-uml-class-diagram}.

It can be seen that a class is represented by a rectangle
which is divided into three parts:
\begin{itemize}
\item
The top part contains the name of the class.
\item
The middle part contains the attributes of the class.
\item
The bottom part gives the methods or operations the class can take or undertake.
\end{itemize}

\begin{figure}[h!]
\centering
\begin{myuml}
  \begin{class}[text width=8cm]{Dog}{0,0}
    \attribute{-breed : String}
    \attribute{-name : String}
    \operation{+print() : void}
  \end{class}
\end{myuml}
\caption{UML class diagram of the \protect\code{Dog} class.}
\label{figure:scs-computer-model->dog-uml-class-diagram}
\end{figure}






\subsection{Software Design Patterns}

In the implementation of \scs,
we make use of \emph{software design patterns}.
Wikipedia explains what are software design patterns
and why they are used:
\begin{quotation}
In software engineering,
a design pattern is a general reusable solution
to a commonly occurring problem within a given context in software design.
A design pattern is a description or template
for how to solve a problem that can be used in many different situations.
Patterns are formalized best practices
that the programmer can use to solve common problems
when designing an application or system.
\end{quotation}

For more information about software design patterns,
refer to the seminal book on design patterns: \cite{gang-of-four-1994}.





\subsubsection{Builder design pattern}

The \emph{builder design pattern}
is useful when the construction of an object requires many parameters.
The builder design pattern uses a builder object
which receives each initialization parameter step by step
and then returns the resulting constructed object at once.

We illustrate the use of the builder design pattern using a simple example
of a \code{Person} class,
which has fields \code{firstName}, \code{lastName} and \code{salutation}.

\begin{lstlisting}
public class Person {
    private String firstName;
    private String lastName;
    private String salutation;
}
\end{lstlisting}

The traditional Java constructor approach might define a constructor
\begin{lstlisting}
Person(String firstName, String lastName, String salutation)
\end{lstlisting}
However, if a user is not careful,
the user may call the constructor with the wrong order of parameters:
\begin{lstlisting}
Person person = new Person("Barack", "Obama", "President") // right!
Person person = new Person("President", "Barack", "Obama") // wrong!
\end{lstlisting}

In contrast, a builder object can be used
to instantiate a \code{Person} object as follows:
\begin{lstlisting}
Person person = new Person.Builder()
        .withFirstName("Barack")
        .withLastName("Obama")
        .withSalutation("President")
        .build();
\end{lstlisting}

As the previous code snippets show,
using the builder design pattern makes the code more readable
and less error-prone.

\scs\ uses the builder design pattern to create objects
which require many parameters in their creation,
e.g.\ the classes \code{Shipment}, \code{IstockSimulator}, among others.
For instance, the code to create a \code{Shipment} object
is shown in \autoref{figure:scs->computer-model->create-a-Shipment-object}.

\begin{figure}[h!]
\begin{lstlisting}
// create a new shipment object
Shipment shipment = new Shipment.Builder()
.withPeriodReceived(period + thisLeadTime.getLeadTime(to.getId(), period))
.withPeriodSent(period)
.withQuantity(quantity)
.withTo(to)
.build();
\end{lstlisting}
\caption{Code to create a \code{Shipment} object,
which uses the builder design pattern.}
\label{figure:scs->computer-model->create-a-Shipment-object}
\end{figure}
created using.





\subsubsection{Abstract factory design pattern}

The \emph{abstract factory design pattern} is a creational design pattern.
As defined by \cite{gang-of-four-1994},
the goal of using the abstract factory design pattern is to
``provides an interface for creating families
of related or dependent objects without specifying their concrete classes.''

We illustrate the abstract factory design pattern using the following example.

Suppose that we were writing
a graphical design application for drawing shapes.
The abstract \code{Shape} class
defines the \code{draw} and \code{move} methods
which must be implemented by its concrete subclasses:
\code{Circle}, \code{Triangle} and \code{Rectangle}.
We can define an abstract \code{ShapeFactory} class
with an abstract method \code{createShape}.
Then we can define subclasses of the \code{ShapeFactory} class
as factories for each of the shapes:
\code{CircleFactory}, \code{TriangleFactory}, \code{RectangleFactory}.
By using the abstract factory design pattern,
we can use the same code
in the source code of the graphical design application
for drawing shapes.
When the user specifies which shape to draw,
the user interface object can create
a corresponding \code{ShapeFactory} object
which is passed to the graphical design object.
The code in the controller object might look like the following:
\begin{lstlisting}
ShapeFactory shapeFactory = 
        userInterface.getShapeFactoryFromUser();
graphicalDesign.drawShape(shapeFactory.createShape());
\end{lstlisting}

\scs\ uses the abstract factory design pattern
in the controller objects
(see \autoref{section:scs->computer-model->controller-objects})
to create the demand model object and lead time model objects.





\subsubsection{Repository Design Pattern}

Using the terminology of \cite{evans-2004},
the \emph{repository design pattern}
is used to create an abstraction between the domain and data layer.
We do not define what is a domain layer or a data layer.
(We refer interested readers to the book.)
Instead, to illustrate how the repository design pattern works,
we consider the example of a shipment repository.
The \code{ShipmentRepository} class defines an object
which acts as a repository for all \code{Shipment} objects,
i.e.\ all \code{Shipment} objects
are stored within the \code{ShipmentRepository} class.
The \code{ShipmentRepository} class has methods
both for adding new \code{Shipment} objects into the repository,
and for removing all \code{Shipment} objects matching certain criteria.
Therefore, the \code{ShipmentRepository} class
hides the internal details of how it manages \code{Shipment} objects,
and allows other classes to use
the interface defined by the \code{ShipmentRepository} class
to add, access or remove \code{Shipment} objects.
This is advantageous because it encapsulates all the code
related to the storage and retrieval of \code{Shipment} objects
in a single \code{ShipmentRepository} class,
which we can be test systematically.

\scs\ uses repository objects to manage
facilities (\code{FacilityRepository}),
shipments (\code{ShipmentRepository})
and reports (\code{ReportRepository}).




\section{Controller Objects}
\label{section:scs->computer-model->controller-objects}

In an OOP paradigm,
every domain object (entity in the conceptual model of the supply chain)
is represented by a software object in the computer code.
Therefore, we define software objects such as
\code{RetailFacility}, \code{Shipment} and \code{Report}
to model entities in the domain which is a supply chain.
We use \emph{controller objects}
in order to coordinate the interaction
of the domain objects in the supply chain
and to implement the control logic specified in the Conceptual Model chapter.
For example, the controller object
gets a \code{Report} object from the customer facility object,
and arranges for the \code{Report} object
to be received by the supplier facility
after the \code{Report} lead time specified by the
\code{ReplenishmentPolicy} object.

\begin{figure}[h!]
\centering
\begin{myuml}
  \begin{abstractclass}{AbstractSimulator}{0,0}
  \end{abstractclass}

  \begin{class}{IstockSimulator}{-3,-2}
    \inherit{AbstractSimulator}
  \end{class}

  \begin{class}{XdockSimulator}{3,-2}
    \inherit{AbstractSimulator}
  \end{class}
\end{myuml}
\caption{UML diagram illustrating
the class hierarchy of the controller objects.}
\label{figure:scs->computer-model->controller-objects}
\end{figure}

The class \code{AbstractSimulator} is a controller object for \scs.
The class \code{AbstractSimulator} is an abstract class,
and is a superclass of the two classes
\code{IstockSimulator} and \code{XdockSimulator}.
As suggested by their names,
\code{IstockSimulator} is a controller object
for a supply chain that operates in an intermediate stocking configuration,
whereas \code{XdockSimulator} is a controller object
for a supply chain that operates in a cross-docking configuration.
The purpose of the \code{AbstractSimulator} class
is to contain the definition of methods
that are common to both its child classes.
The class hierarchy is illstrated in
\autoref{figure:scs->computer-model->controller-objects}.

The conceptual model of the sequence of events in a period
is defined in \autoref{section:conceptual-model->sequence-of-events},
The sequence of events is implemented in the computer model
in the \code{runSimulation} method
of the class \code{AbstractSimulator}
(see \autoref{figure:scs->computer-model->AbstractSimulator}).
The methods \code{submitReports} and \code{receiveAndSendShipments}
of the \code{AbstractSimulator} class are abstract.
The subclasses \code{IstockSimulator} and \code{XdockSimulator}
implement these abstract methods
to implement the behavior of an
intermediate stocking and cross-docking configuration respectively.

\begin{figure}[h!]
\begin{lstlisting}
public abstract class AbstractSimulator {
    /** Run the simulation. */
    public void runSimulation(SimulationParameters simulationParameters)
            throws Exception {
        // create facility, demand and lead time objects
        createObjects();

        for (int t = simulationStartPeriod; t < simulationEndPeriod; ++t) {
            // for each facility, log the beginning of period inventory level
            logStartOfPeriodInventory(t);
            // for each facility, submit a report to its supplier according
            // to the replenishment policy
            submitReports(t);
            // for each facility, from national to regional to retail,
            // receive shipments that are due to arrive,
            // make shipments if applicable
            receiveAndSendShipments(t);
            // demand arrives at the facility
            demandArrives(t);
        }
    }

    protected abstract void submitReports(int t);

    protected abstract void receiveAndSendShipments(int t);
}
\end{lstlisting}
\caption{Abbreviated contents of the \protect\code{AbstractSimulator} class.}
\label{figure:scs->computer-model->AbstractSimulator}
\end{figure}





\section{Supply Chain Topology}
\label{section:scs->computer-model->supply-chain-topology}


In the conceptual model of the supply chain,
the supply chain has three tiers,
a single national facility,
one or more regional facilities,
and one or more retail facilities.
The three facility tiers are represented
in the computer model of the supply chain,
by the three classes:
\code{NationalFacility}, \code{RegionalFacility} and \code{RetailFacility}.
Internally, each of the facility classes
is a subclass of the parent class \code{Facility}.
This is illustrated in
\autoref{figure:scs->computer-model->facilities-uml}.

\begin{figure}[h!]
\centering
\begin{myuml}
  \begin{abstractclass}[text width=8cm]{Facility}{0,0}
    \attribute{-id : String}
    \attribute{-inventory : int}
    \attribute{-demandArray : NegativeArray$<$int$>$}
    \attribute{-unmetDemandArray : NegativeArray$<$int$>$}
    \attribute{-consumptionArray : NegativeArray$<$int$>$}
    \attribute{-inventoryArray : NegativeArray$<$int$>$}
    \attribute{-shipmentArray : NegativeArray$<$int$>$}
    \operation{+getReportBuilder(t : int, h : int) : Report.Builder}
    \operation{+logStartOfPeriodInventory(t : int) : void}
    \operation{+demandArrives(t : int, d : int) : void}
    \operation{+get and set array methods}
  \end{abstractclass}

  \begin{class}[text width=3cm]{NationalFacility}{-4,-7}
    \inherit{Facility}
  \end{class}

  \begin{class}[text width=3cm]{RegionalFacility}{0,-7}
    \inherit{Facility}
  \end{class}

  \begin{class}[text width=3cm]{RetailFacility}{4,-7}
    \inherit{Facility}
  \end{class}
\end{myuml}
\caption{UML diagram illustrating the implementation of the facility classes.}
\label{figure:scs->computer-model->facilities-uml}
\end{figure}


The \code{Facility} class has the following fields:
\begin{itemize}
\item \code{id}
A string that is a unique identifier for the facility.
\item \code{inventory}
The current inventory level at the facility
\item \code{\{demand,unmetDemand,consumption,inventory,shipment\}Array}
A record of the demand, unmet demand, consumption,
beginning of period inventory level,
and shipment received in each period respectively.
\end{itemize}


The supply chain topology
consists of the set of facilities
and the assignment of each retail facility to a regional facility.
This is modeled in the computer model
by the class \code{Topology}.





\section{Exogenous Supply and Demand}
\label{section:scs->computer-model->supply-and-demand}

The national supply schedule is modeled by
the \code{NationalSupplySchedule} object.
Conceptually, the current implementation
of the \code{NationalSupplySchedule} object
is a cyclic supply schedule,
i.e.\ a constant quantity is delivered to the national facility
every $C$ periods where $C$ is the cycle length.

The exogenous demand is modeled by the \code{DemandModel} object.
Conceptually, the \code{DemandModel} object
generates the random demand quantity
for each retail facility and for each period in the simulation horizon.





\section{Inventory Replenishment Process}
\label{section:scs->computer-model->replenishment-process}

A conceptual model of the inventory replenishment process
was presented in \autoref{section:conceptual-model->inventory-replenishment}.
By way of reminder,
tor each (supplier, customer) relationship,
the inventory replenishment process is the following:
\begin{enumerate}
\item
The customer facility sends a report to its supplier facility.
\item
The supplier facility receives the report,
and computes a replenishment quantity
based on the inventory replenishment policy,
the report received,
and the current inventory level of the supplier facility.
\item
The shipment arrives at the customer facility.
\end{enumerate}

We describe below
how each of the three steps are implemented in the computer code.





\subsection{Implementation of Reports}

\paragraph{How a customer facility submits a report}
The \code{Facility} class has a method \code{getReportBuilder}
which returns a \code{Report.Builder} object.
The \code{AbstractSimulator} uses this object
to build a \code{Report} object,
which it inserts into a \code{Report} repository.

\paragraph{How a supplier facility receives a report}
The implementation of a supplier facility receiving a report
is done by the \code{AbstractReplenishmentPolicy} class.
The class is described in detail below.
For the time being,
we note that the method \code{getShipmentDecisions}
of the \code{AbstractReplenishmentPolicy} class
removes reports that are due to arrive at the
supplier facility at the current time period
from the \code{ReportRepository},
and processes the reports.





\subsection{Implementation of Inventory Replenishment Policies}

An inventory replenishment policy is implemented in the computer code
by the class \code{AbstractReplenishmentPolicy} which is an abstract class.
The software architecture is shown in
\autoref{figure:scs->computer-model->uml->replenishment-policy}.
The class has two fields.
\begin{enumerate}
\item
The field \code{oneTierReportDelay}
which represents the number of periods of delay
between a facility sending a report
and the facility immediately upstream receiving the report.
\item
The field \code{function}
is an \code{AbstractReplenishmentFunction},
which is a function mapping the inputs
(supplier facility inventory level,
the received reports,
the pipeline shipments,
the demand forecast,
the shipment schedule,
and the lead time forecast)
into a list of shipment decisions.
\end{enumerate}

The class \code{AbstractReplenishmentPolicy}
is a class that wraps around an \code{Abstract\-Replenishment\-Function}.
So if we put an order-up-to replenishment function
into the \code{Abstract\-Replenishment\-Policy} class,
we would get an order-up-to replenishment policy;
and if we put an constant replenishment function
into the \code{AbstractReplenishmentPolicy} class,
we would get a constant replenishment policy.

\begin{figure}[h!]
\centering
\begin{myuml}
  \begin{abstractclass}[text width=10cm]{AbstractReplenishmentPolicy}{0,0}
    \attribute{-oneTierReportDelay : int}
    \attribute{-function : AbstractReplenishmentFunction}
    \operation{getReportDelay() : int}
    \operation{getReportHistoryPeriods() : int}
    \operation{getShipmentDecisions(input : ReplenishmentFunctionInput)
      : Collection<ShipmentDecision>}
  \end{abstractclass}

  \begin{abstractclass}[text width=10cm]{AbstractReplenishmentFunction}{0,-6}
    \operation{getReportHistoryPeriods() : int}
    \operation{getShipmentDecisions(input : ReplenishmentFunctionInput)
      : Collection<ShipmentDecision>}
  \end{abstractclass}
  
  \composition{AbstractReplenishmentPolicy}{function}{1..1}
  {AbstractReplenishmentFunction}
\end{myuml}
\caption{UML class diagram illustrating how 
the \protect\code{AbstractReplenishmentPolicy} class
is a wrapper class around
the \protect\code{AbstractReplenishmentFunction} class.}
\label{figure:scs->computer-model->uml->replenishment-policy}
\end{figure}





The replenishment policy for a generic (supplier, customer) relationship
is modeled by the \code{AbstractReplenishmentPolicy} class.
This class has two subclasses:
\code{AbstractOneTierReplenishmentPolicy}
for (national facility, regional facility)
and (regional facility, retail facility) replenishment
in an intermediate stocking configuration;
and \code{AbstractXdockReplenishmentPolicy}
for (national facility, retail facility) replenishment
in a cross-docking configuration.
This is illustrated in
\autoref{figure:scs->computer-model->uml->replenishment-policy-hierarchy}.

\begin{figure}[h!]
\centering
\begin{myuml}
  \begin{abstractclass}{AbstractReplenishmentPolicy}{0,0}
  \end{abstractclass}

  \begin{class}{IstockReplenishmentPolicy}{-3,-2}
    \inherit{AbstractReplenishmentPolicy}
  \end{class}

  \begin{class}{XdockReplenishmentPolicy}{3,-2}
    \inherit{AbstractReplenishmentPolicy}
  \end{class}
\end{myuml}
\caption{UML class diagram illustrating the class hierarchy of 
the \protect\code{AbstractReplenishmentPolicy} class
and its subclasses.}
\label{figure:scs->computer-model->uml->replenishment-policy-hierarchy}
\end{figure}





\subsection{Implementation of Shipments}

\paragraph{How a supplier facility sends a shipment}
The class \code{Abstract\-Replenishment\-Policy}
has a method \code{getShipmentDecisions},
which returns a list of shipment decisions
(i.e.\ the shipment quantity and the destination customer facility).
The class \code{AbstractSimulator}
has a method \code{makeShipment}
which executes this shipment decision.
First, the method instantiates a \code{Shipment} object.
Next, the method ``sends'' the shipment
by adding the \code{Shipment} object into the \code{ShipmentRepository}.

\paragraph{How a customer facility receives a shipment}
The implementation of a customer facility receiving a shipment
is found in the abstract method \code{receiveAndSendShipments}
of the class \code{AbstractSimulator}.
The overridden implementation of this method
in the subclasses \code{IstockSimulator} and \code{XdockSimulator}
use the method \code{receiveShipments}
of the class \code{AbstractSimulator}.
Basically, the method removes all \code{Shipment} objects
that are due to arrive at the customer facility at the current period
from the \code{ShipmentRepository},
and increments the inventory level at the customer facility
by the total inventory quantity in those \code{Shipment} objects.





\section{Design of Computer Experiments}
\label{section:scs->computer-model->design-of-experiments}

One of the challenges in managing the \scs\ supply chain
is that the demand and shipment lead times are random.
In this section,
we describe how the random variates
for a particular simulation replication are generated in the computer model.
We also discuss how to choose simulation run parameters
such as the initial inventory level and the simulation run horizon.






\subsection{Random Variate Generation}

We chose to implement the random objects \code{Demand} and \code{LeadTime}
as \emph{immutable} objects
\citep{wikipedia->immutable-object},
i.e.\ its state cannot be modified after it is created.
This reduces programming errors because the random variates
cannot be modified accidentally after the object is created.

Another design principle for the random objects
are that they are reproducible,
i.e.\ for the same set of facilities,
the same start and end periods,
and the same random seed,
the generated random variates are the same.
This is advantageous from a debugging perspective,
but also useful as a variance reduction technique.

The goal of \scs\
is to compare inventory replenishment policies through simulation.
In order to increase the statistical precision of the estimates
that can be obtained for a given number of iterations,
we employ a variance reduction procedure
where the demands and lead times
in replication $i$ of a particular inventory policy
are equal to the demands and lead times
in replication $i$ of another inventory policy.





\subsection{Initial Conditions}

Initial inventory levels at facilities.
For intermediate stocking configuration,
the initial inventory level at the regional facilities and the retail facilities.
For cross-docking configuration,
the initial inventory level at the retail facilities.
The level is specified as an order-up-to level.
\todo{More details as I implement the order-up-to level object.}






\subsection{Simulation Horizon}

The goal of the simulation is to estimate
the long-run performance of an inventory replenishment policy.
In other words, when the supply chain system is in steady state,
what is the expected service level, or expected average inventory level?

In order to collect data when the supply chain system is in steady-state,
it is necessary to run the simulation for a warm-up period
before beginning the data collection period.

The \emph{warm-up period} is the number of periods
that the simulation will run before the data collection period.
The purpose of the simulation warm-up period
is for the system to transition
from the initial transient conditions
to steady-state conditions.

The \emph{data collection period} is the number of periods
that the simulation will run and data will be collected.

The simulation start and end period is specified
by the \code{SimulationParameters} object.
Statistics of interest are obtained from the
\code{SimulationResults} object
which is created based on an \code{AbstractSimulator} object.
To obtain an estimate of a statistic of interest
based only on data during the data collection period,
use a method while indicating the start and end periods
of the data collection period, e.g.
\begin{lstlisting}
AbstractSimulator simulator = // IstockSimulator or XdockSimulator
simulator.runSimulation(...);
SimulationResults results = new SimulationResults(simulator);
int startPeriod = 0;
int endPeriod = 10;
results.getServiceLevel(startPeriod, endPeriod);
\end{lstlisting}

