\chapter[Conceptual Model (Zambia)]{Conceptual Model of Zambia Supply Chain}

In this chapter, we describe our conceptual model
for the Zambia public-sector supply chain for medical drugs.
This is a very brief description of the Zambia supply chain
for the purposes of showing how to use the supply chain simulator.
For a detailed description,
we refer the reader to \citet{gallien-leung-yadav-2014}.
We focus our attention on modeling the demand for a single drug,
ACT (Artemisinin-based Combination Therapy),
which is the standard and most effective first-line medicine for malaria.





\section{Discrete Time Framework}
\label{section:zambia->computer-model->time-unit}

For the conceptual model of the Zambia supply chain,
we chose to use a period as the basic time unit,
with a period defined as 1/48-th of a year,
or equivalently for one year to be composed of 48 periods.
This is convenient for several reasons:
\begin{enumerate}
\item
The demand for antimalarial drugs has an annual seasonality,
because climatic factors (particularly rainfall)
influence the incidence of malaira.
By using the basic time unit of a period equal to 1/48-th of a year,
the demand mean for period $0$ is equal to the demand mean in period $48$.
If we had chosen the basic time unit of a period equal to a week,
then a year is composed of $52\frac{1}{7}$ periods,
so that the demand mean for period $0$
is not equal to the demand mean in period $52$.
\item
The inventory replenishment process in Zambia
occurs according to a monthly cycle.
By using the basic time unit of a period equal to 1/48-th of a year,
the schedule of events in the replenishment process
occurs on regular period numbers,
e.g.\ the shipment periods to a facility might be periods $0, 4, 8, 12$.
If we had chosen the basic time unit of a period equal to a week,
then the shipment period to a facility would be irregular,
e.g.\ the shipment periods to a facility might be periods $0, 5, 9, 13, 18$.
\item
For the purpose of computing monthly statistics,
e.g.\ average service level in January,
it is easier to convert periods to calendar months
when we use the basic time unit of a period equal to 1/48-th of a year.
\end{enumerate}






\section{Supply Chain Topology}

The Zambia public-sector supply chain for medical drugs has three tiers.
\begin{itemize}
\item
The first tier is the national warehouse in the captial city, Lusaka.
Drugs entering the country first arrive at the national warehouse.
The national warehouse is operated by a para-statal company
known as Medical Stores Limited (MSL).
\item
The second tier facilities are the district warehouses.
Politically, Zambia is divided into ten provinces,
which are further subdivided into 89 districts.
\item
Finally, the third tier is composed of approximately 1500 health facilities.
\end{itemize}




\section{Exogenous Supply and Demand}

We assume that MSL is given a predictable schedule
of inventory deliveries at the national warehouse.

We focus our attention on modeling the demand for a single drug,
ACT (Artemisinin-Based Combination Therapy),
which is the standard and most effective first-line medicine for malaria.
It is recommended that a malaria patient
should receive this treatment within 24 hours of starting a fever.
If the patient does not receive ACT drugs promptly,
then the patient could progress to severe malaria
which requires more aggressive medical treatment.
Based on these characteristics,
we chose to model the Zambia supply chain as a lost sales system.

We propose to use a multiplicative
martingale model of forecast evolution
\citep{heath-jackson-1994}
to generate the demand.
We chose this demand model
because it is able to generate both demands
and demand forecasts
which are useful for the optimization-based replenishment policy
but also more sophisticated forecast-based replenishment policies.





\section{Inventory Replenishment Process}

\subsection{Inventory Replenishment Constraints}

The inventory replenishment process in Zambia
occurs on a monthly cycle.
In other words, at a particular date every month,
a customer facility submits an order form to its supplier facility,
and at another date every month,
the supplier facility packs and sends a shipment to the customer facility.

The existing order process in Zambia relies on paper order forms.
We assume that there is a one week delay
for a paper form to travel from a customer facility to a supplier facility.
There has been a proposal to replace paper order forms by cell phones.
We assume that using cell phones to keep inventory records
would allow orders to be transmitted instantaneously.

In the Zambia supply chain,
the primary distribution of medical drugs
(i.e.\ the national warehouse making shipments to district warehouses)
occurs according to a monthly shipment schedule that is pre-determined by MSL.
The inventory is transported by ten-tonne trucks
which travel on main arterial highways of Zambia.
Consequently, primary distribution is both regular and predictable.

In contrast,
the secondary distribution of medical drugs
(i.e.\ district warehouses making shipments to health facilities)
is irregular due to resource constraints at the district level.
The roads to the more rural health facilities
are primitive gravel or dirt road surfaces.
The two most important transporation constraints are:
\begin{enumerate}
\item
There is a lack of a dedicated vehicle for drug delivery at the district.
Any district vehicles are shared between multiple health programs.
Thus in practice, drug delivery to a given health facility
can only occur if a district vehicle happens to visit that health facility.
\item
For certain health facilities at certain times of the year,
the health facility may be rendered temporarily inaccessibile
due to the roads leading to the facility being temporarily flooded.
\end{enumerate}
Based on these two secondary shipment constraints,
we propose a lead time model based on two Bernoulli random variables,
one for a vehicle from the district being available
to make a visit to a health facility in a given period,
and another for the road to the health facility being accessibile.
A shipment only takes place if
both of these Bernoulli random variables are successes.

As in the \scs\ supply chain,
the Zambia supply chain can operate in
either an intermediate stocking configuration
or a cross-docking configuration.

\paragraph{Timing of replenishment processes}
For primary distribution,
each district warehouse is required to submit its order form
to the national warehouse by a scheduled cutoff date.
In an intermediate stocking configuration,
the district warehouse must submit its order form by the order cutoff;
whereas the health facilities are scheduled to submit their order forms
on the first day of each month.
In a cross-docking configuration,
each health facility is schedule to submit its order form
to the national facility
by the scheduled cutoff date for orders from
the district warehouse corresponding to the health facility.






\subsection{Existing Inventory Replenishment Policies}

During the Zambia supply chain pilot in 2009/2010,
two inventory replenishment policies were tested.
Both policies were versions of \emph{order-up-to policies},
in other words the replenishment quantity is
the amount of inventory needed
to bring the inventory position
up to an order-up-to level:
\[
    \text{replenishment quantity}
  =
    \text{order-up-to level}
    - \text{inventory position}.
\]
In the Zambia replenishment policies,
the order-up-to level for a given facility
is a multiple $M$ of the average monthly issues (AMI),
which is the average monthly quantity of the drug
issued over the last three months at the facility;
and the inventory position is defined as the
sum of the current inventory level
and the shipment quantities that are currently in transit to the facility.

The two replenishment policies that were tested in Zambia are:
\begin{enumerate}
\item
A intermediate stocking policy
with an order-up-to level of three months AMI at the district facilities
and an order-up-to level of two months AMI at the health facilities.
\item
A cross-docking policy
with an order-up-to level of four months AMI at the health facilities.
\end{enumerate}






\subsection{Optimization-Based Replenishment Policy}
\label{subsection:zambia->conceptual-model->optimization-policy}

As researchers in the field of supply chain management,
we proposed a cross-docking optimization-based inventory replenishment policy,
which takes into account the following factors:
\begin{itemize}
\item the schedule of deliveries at the national warehouse
\item the primary distribution shipment schedule
\item the probability distribution of shipment lead times
\item the forecast for future demand
\item the inventory level at the health facilities
\item the shipments in transit to the health facilities
\end{itemize}

For a mathematical definition of the optimization-based
inventory replenishment policy,
see \citet{gallien-leung-yadav-2014}.





\subsection{Clairvoyant Replenishment Policy}

In order to benchmark the performance of inventory replenishment policies,
we propose a clairvoyant replenishment policy,
which is able to foresee future demands and lead times with 100\% accuracy.
Clearly, this policy cannot be used in practice.
Nevertheless it is useful as an upper bound to the performance
of any replenishment policy.
The clairvoyant replenishment policy is an optimization-based policy
in a similar spirit to the optimization-based policy
described in \autoref{subsection:zambia->conceptual-model->optimization-policy},
except that we replace the lead time percentiles by the realized lead time,
and the demand tangents by the realized demands.

