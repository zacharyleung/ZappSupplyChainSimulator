\chapter[Computer Model (Zambia)]{Computer Model of Zambia Supply Chain}

We build a computer model of the Zambia supply chain using \scs,
by mapping the entities in the Zambia supply chain
to entities in \scs\ supply chain.


We explain our choice of time unit in
\autoref{section:zambia->computer-model->time-unit},
we explain the format of the input files in section blah,
we explain which classes implement the replenishment policies in section blah.




\section{Mapping Entities to Classes}

The mapping of entities in the conceptual model of the Zambia supply chain
to classes in the computer model of the \scs\ supply chain
is shown in \autoref{table:zambia:computer:zambia-to-scs-mapping}.

\begin{table}[h!]
\centering
\begin{tabular}{ll}
\toprule
  Zambia Entity & \scs\ Class \\
\midrule
  National warehouse & \code{NationalFacility} \\
  District warehouse & \code{RegionalFacility} \\
  Health facility & \code{RetailFacility} \\
  Order form & \code{Report} \\
  Order-up-to replenishment policy & \code{OrderUpToReplenishmentPolicy} \\
  Optimization & \code{GlyReplenishmentPolicy} \\
  Clairvoyant replenishment policy & \code{ClairvoyantReplenishmentPolicy} \\
  Vehicle accessibility lead time & \code{VehicleAccessiblityLeadTime} \\
  Multiplicative MMFE demand model & \code{MmfeDemand} \\
\bottomrule
\end{tabular}
\caption{A mapping between the entity in the Zambia supply chain
and the \scs\ class representing that entity.}
\label{table:zambia:computer:zambia-to-scs-mapping}
\end{table}

Both the Zambia supply chain and the \scs\ supply chain have three tiers.
Thus, there exists a natural mapping
from facilities in the Zambia supply chain
to facilities in the \scs\ supply chain.



\section{Input Files}

The input files for the Zambia use case are found in the folder
\begin{quote}
\code{SupplyChainSimulator/input/zambia/}
\end{quote}
The input files are stored as comma-separated values (CSV) files,
i.e.\ tabular data stored as a plain text file,
with each field separated by a comma.




\subsection{Supply Chain Topology}

The supply chain topology for Zambia
is defined in the input file
\begin{quote}
\code{SupplyChainSimulator/input/zambia/health-facilities.csv}
\end{quote}
\begin{figure}[h!]
\begin{lstlisting}
regional,retail
chama,buli
chama,chibale
chama,chifunda
\end{lstlisting}
\caption{Partial listing of the contents of
the file \protect\code{health-facilities.csv}.}
\label{figure:zambia->health-facilities.csv}
\end{figure}


The first four lines of the file are shown
in \autoref{figure:zambia->health-facilities.csv}.
The first line is a header line.
The following lines are (regional facility, retail facility) pairs.
For instance, the second line means that a retail facility named ``Buli''
is located in a region named ``Chama.''
The Zambia use case has 212 health facilities
located in 12 districts.









\subsection{Demand Means}

We described a procedure for estimating the mean demand at
212 health facilities in Zambia
in \citet{gallien-leung-yadav-2014}.
The estimated demand means are listed in the input file
\begin{quote}
\code{SupplyChainSimulator/input/zambia/demand-mean.csv}
\end{quote}
As explained in \autoref{section:zambia->computer-model->time-unit},
for the purposes of simulation,
we chose to use a time unit where one period is equal to 1/48-th of a year.
Thus, the input file defines the mean demand
for each health facility for each period of the year.

\begin{figure}[h!]
\begin{lstlisting}
retail,1,2,3,4,5,6,7,...
buli,44.20,48.00,50.10,49.40,46.10,40.90,35.20,...
chibale,133.00,152.00,171.00,186.00,196.00,201.00,202.00,199.00,...
chifunda,47.10,58.00,67.30,72.60,73.90,72.20,69.30,66.70,...
\end{lstlisting}
\caption{Partial listing of the contents of
the file \protect\code{demand-mean.csv}.}
\label{figure:zambia->demand-mean.csv}
\end{figure}

The first four lines and first eight columns of the file are shown
in \autoref{figure:zambia->demand-mean.csv}.
The first line is a header line.
The following lines are entries of the form
(retail facility $r$, $d_{r1}, d_{r2}, \ldots$).
For instance, the second line means that a retail facility named ``Buli''
has mean demand in period 1 of 44.20,
has mean demand in period 2 of 48.00,
etc.




\subsection{Replenishment Schedule and Lead Times}

The primary shipment schedule,
primary and secondary shipment lead times
are defined in the input file
\begin{quote}
\code{SupplyChainSimulator/input/zambia/replenishment.csv}
\end{quote}

\begin{figure}[h!]
\begin{lstlisting}
regional,shipment offset,primary lead time,mean secondary lead time
chama,0,3,2.308
chavuma,0,3,2.055
kabompo,0,3,3.621
kaoma,1,2,3.727
\end{lstlisting}
\caption{Partial listing of the contents of
the file \protect\code{replenishment.csv}.}
\label{figure:zambia->replenishment.csv}
\end{figure}

The first five lines of the file are shown in
\autoref{figure:zambia->replenishment.csv}.
Each line in the input file contains 4 columns,
corresponding to (regional facility, shipment offset,
primary lead time, mean secondary lead time).
\begin{itemize}
\item
The \emph{shipment offset} is an integer in the set $\{0,1,2,3\}$.
If the shipment offset for a regional facility is $b$,
then the national facility is scheduled to make a shipment
to that regional facility in periods $t \equiv \amodb{b}{4}$.
For example, line 2 indicates that the national facility makes shipments
to the Chama regional facility
during periods $\{0, 4, 8, \ldots\}$;
while line 5 indicates that the national facility makes shipments
to the Kaoma regional facility
during periods $\{1, 5, 9, \ldots\}$.
\item
The \emph{primary lead time} is an integer,
which represents the number of periods between
the national facility making a shipment to a regional facility
and the regional facility receiving the shipment.
\item
The \emph{mean secondary lead time} is a real number,
which represents the mean number of periods between
the regional facility making a shipment to a retail facility in its region
and the retail facility receiving the shipment,
assuming that the retail facility always has 100\% accessibility probability.
\end{itemize}





\subsection{Health Facility Accessibility}

The health facility accessibility probabilities
are defined in the input file
\begin{quote}
\code{SupplyChainSimulator/input/zambia/health-facilities-accessibility.csv}
\end{quote}

\begin{figure}[h!]
\begin{lstlisting}
Health Facility,AccessibilityJan,AccessibilityFeb,...,AccessibilityDec
buli,0,0,0,0.75,1,1,1,1,1,1,1,0.5
chibale,1,1,1,1,1,1,1,1,1,1,1,1
chifunda,0,0,0,0,0.5,1,1,1,1,1,1,0.5
\end{lstlisting}
\caption{Partial listing of the contents of
the file \protect\code{health-facilities-accessibility.csv}.}
\label{figure:zambia->health-facilities-accessibility.csv}
\end{figure}

The first four lines of the file are shown in
\autoref{figure:zambia->health-facilities-accessibility.csv}.
Each line in the input file contains 13 columns.
The first column is the name of the health facility.
The next twelve columns are the probabilities
that the health facility is accessible by a four-wheel drive vehicle
during the particular month of the year.





\section{Code Walkthrough}

In this section,
we walk through some of the computer code
which may be of interest to researchers in the field of inventory management.
In particular, we go over the implementation of
inventory replenishment policies,
demand models and lead time models.



\subsection{Order-Up-To Replenishment Policies}

Conceptually, there are two different versions
of order-up-to replenishment policies:
an intermediate stocking order-up-to replenishment policy,
and a cross-docking order-up-to replenishment policy.
For both these versions,
the calculations of shipment quantities
is implemented in the class \code{Order\-Up\-To\-Replenishment\-Function}.

We now walk through the code of the class
\code{Order\-Up\-To\-Replenishment\-Function},
which is shown in
\autoref{figure:zambia->computer-model->OrderUpToReplenishmentFunction}.
The computation of the order-up-to level depends on two input parameters,
which are fields of the class \code{Order\-Up\-To\-Replenishment\-Function}:
\begin{itemize}
\item
\code{historyPeriods}
The number of periods of demand/consumption
for which to compute the average monthly demand/consumption.
\item
\code{orderUpToPeriods}
The order-up-to level is this multiple
of the average monthly demand/consumption.
\end{itemize}
The steps to calculate the shipment quantity are:
\begin{enumerate}
\item Calculate the mean demand per period
\item Calculate the order-up-to level
\item Calculate the inventory position
\item Calculate the shipment quantity
\end{enumerate}
The reader can see how the code in
\autoref{figure:zambia->computer-model->OrderUpToReplenishmentFunction}
implements these steps to compute the shipment quantity.


\begin{figure}[h!]
\begin{lstlisting}
LinkedList<ShipmentDecision> shipmentDecisions = new LinkedList<>();
int inv = supplierFacility.getInventory();

for (Report report : reports) {
    // 1. Calculate the mean demand per period
    double sum = 0;
    for (int i = 0; i < historyPeriods; ++i) {
        sum += report.getDemand(i);
    }
    double d = sum / historyPeriods;
    
    // 2. Calculate the order-up-to level
    double out = d * orderUpToPeriods;
    
    // 3. Calculate the inventory position
    int ip = report.getInventory();
    for (Shipment shipment : report.getShipments()) {
        ip += shipment.quantity;
    }
    
    // 4. Calculate the shipment quantity
    int q = (int) (out - ip);
    // take the min of the intended quantity and the supplier's
    // remaining inventory level
    q = Math.min(q, inv);
    inv -= q;
    
    // only create a shipment if the shipment quantity > 0
    if (q > 0) {
        ShipmentDecision shipmentDecision = new ShipmentDecision.Builder()
        .withFacility(report.getFacility())
        .withPeriod(currentPeriod)
        .withQuantity(q)
        .build();

        shipmentDecisions.add(shipmentDecision);
    }
}
\end{lstlisting}
\caption{Implementation of the \code{getShipmentDecisions} method in the
\code{OrderUpToReplenishmentFunction} class.}
\label{figure:zambia->computer-model->OrderUpToReplenishmentFunction}
\end{figure}

The code to create an order-up-to intermediate stocking replenishment policy
is shown in
\autoref{figure:zambia->computer-model->create-istock-order-up-to-policies}.
The code to create an order-up-to intermediate stocking replenishment policy
and an order-up-to cross-docking replenishment policy
is shown in
\autoref{figure:zambia->computer-model->create-xdock-order-up-to-policies}.

\begin{figure}[h!]
\begin{lstlisting}
AbstractReplenishmentFunction<Facility, LeadTime>
        replenishmentFunction =
        new OrderUpToReplenishmentFunction
        .Builder<Facility, LeadTime>()
        .withHistoryPeriods(historyPeriods)
        .withOrderUpToPeriods(orderUpToPeriods)
        .build();
IstockReplenishmentPolicy istockReplenishmentPolicy =
        new IstockReplenishmentPolicy(oneTierReportDelay,
                replenishmentFunction);
\end{lstlisting}
\caption{Code to create an order-up-to
intermediate stocking replenishment policy.}
\label{figure:zambia->computer-model->create-istock-order-up-to-policies}
\end{figure}


\begin{figure}[h!]
\begin{lstlisting}
AbstractReplenishmentFunction<NationalFacility, XdockLeadTime> 
        replenishmentFunction =
        new OrderUpToReplenishmentFunction
        .Builder<NationalFacility, XdockLeadTime>()
        .withHistoryPeriods(historyPeriods)
        .withOrderUpToPeriods(orderUpToPeriods)
        .build();
XdockReplenishmentPolicy xdockReplenishmentPolicy =
        new XdockReplenishmentPolicy(oneTierReportDelay,
                replenishmentFunction);
\end{lstlisting}
\caption{Code to create an order-up-to cross-docking replenishment policy.}
\label{figure:zambia->computer-model->create-xdock-order-up-to-policies}
\end{figure}





\clearpage
\subsection{Optimization Replenishment Policy}

The optimization replenishment policy
is a replenishment policy
that is used in a cross-docking supply chain configuration.
The optimization replenishmen policy
is implemented in the class \code{OptimizationReplenishmentPolicy}.
The class \code{OptimizationReplenishmentPolicy}
is a thin class that wraps around
the \code{Optimization\-Replenishment\-Function} class,
which implements the logic for computing shipment quantities.

This is an outline of the method \code{getShipmentDecisions}:
\begin{enumerate}
\item Add the decision variables to the Gurobi model.
For example, the following code creates new decision variables
for the start of period inventory levels
using the method \code{model.addVar},
and stores these decision variables into a \code{HashMap}.
\begin{lstlisting}
HashMap<Key, GRBVar> ir = new HashMap<>();
for (int r = 0; r < R; ++r) {
    for (int s = T0[r]; s <= T2; ++s) {
        ir.put(new Key(r, s), model.addVar(0, GRB.INFINITY, 0.0,
        GRB.CONTINUOUS, String.format("cr[%d][%d]", r, s)));
    }
}
\end{lstlisting}

\item Add the objective function to the Gurobi model.
The following code defines the objective as the sum of
the weighted inventory holding costs and the unmet demand costs
at each health facility.
\begin{lstlisting}
GRBLinExpr obj = new GRBLinExpr();
for (int r = 0; r < R; ++r) {
    double meanAccessibility = 
            MathUtils.getMean(reports[r].getAccessibility()); 
    for (int s = T0[r]; s < T2; ++s) {
        obj.addTerm(meanAccessibility, ir.get(new Key(r, s)));
        obj.addTerm(unmetDemandCost, ur.get(new Key(r, s)));
    }
}
model.setObjective(obj, GRB.MINIMIZE);
\end{lstlisting}

\item Add the constraints one-by-one to the Gurobi model.
For example, the following code declares a new linear constraint
that sets the initial inventory level at the national facility
in the Gurobi optimization model
equal to the current inventory level at the national facility
in the computer simulation model.
\begin{lstlisting}
// i^N_T1 = I^N
constr = new GRBLinExpr();
constr.addTerm(1.0, in.get(T1));
model.addConstr(constr, GRB.EQUAL,
        nationalFacility.getCurrentInventoryLevel(),
        "i^N_T1 = I^N");
\end{lstlisting}

\item Solve the Gurobi model.
To solve the Gurobi model, just call the \code{optimize()} method,
e.g.\ \code{model.optimize()}.

\item Parse the optimal solution of the Gurobi model
to create the shipment decisions.
The following code reads the values
of the shipment decision variables \code{xr}
and executes the shipment decisions
for positive shipment quantities that are to be sent in the current period.
\begin{lstlisting}
for (int r = 0; r < R; ++r) {
    for (int s = T1; s <= T2; ++s) {
        double q = xr.get(new Key(r, s)).get(GRB.DoubleAttr.X);
            if (q > 0.001) {
                if (s == T1) {
                    decisions.add(new ShipmentDecision.Builder()
                    .withFacility(reports[r].getFacility())
                    .withPeriod(s)
                    .withQuantity((int) q)
                    .build());
            }
        }
    }
}
\end{lstlisting}
\end{enumerate}

The following constraints are added to the Gurobi model in step 3.
\begin{enumerate}
\item The inventory balance equations for the national facility
\item The inventory balance equations for each retail facility,
which includes pipeline shipments (shipments already sent)
and future shipments
(shipments which we are deciding in the optimization problem)
\item The tangent constraints which give
a lower bound for the expected unmet demand
\item If the national facility is not scheduled to make a shipment
to a retail facility during a particular period,
set the shipment quantity during that period to $0$
\end{enumerate}

We refer interested readers to \citet{gallien-leung-yadav-2014}
for the details of the optimization formulation.







\subsection{Clairvoyant Replenishment Policy}

The shipment quantity computation for the clairvoyant replenishment policy
is implemented in the class \code{ClairvoyantReplenishmentFunction}.
The implementation is similar to that of
the class \code{OptimizationReplenishmentFunction}.
Likewise, the class \code{ClairvoyantReplenishmentPolicy}
is a wrapper class around the class \code{ClairvoyantReplenishmentFunction}.


\subsection{MMFE Demand Model}

The mathematical definition of the multiplicative MMFE demand model
can be found in \citet{heath-jackson-1994}.
Briefly, the multiplicative MMFE demand model sets
\[
    \rvD_u
  =
    \bar{D}_u \prod_{t=u-M+1}^u \exp(\rvnu_{tu}).
\]
where the parameter $M$ is the number of periods
for which demand information is revealed ahead of the actual period;
and $\rvnu_{tu}$ is a normal random variable
with the standard deviation $\sigma_{u-t}$
and mean $-\sigma_{u-t}^2 / 2$,
so that $\EX{\exp(\rvnu_{tu})} = 1$.
Here $\rvnu_{tu}$ represents the demand uncertainty
regarding the demand in period $u$
that becomes known at the end of period $t$.

The multiplicative MMFE demand model
is implemented in the class \code{Mmfe\-Levels\-Single\-Facility\-Demand},
which inherits from the abstract class
\code{Single\-Facility\-Demand}.
It needs to override two methods defined in the parent class:
\code{getDemand()} and \code{getDemandForecast()}.

The following code in the constructor
generates the random variables $\rvnu_{tu}$
and stores them into a \code{HashMap}:
\begin{lstlisting}
for (int t = startPeriod; t < endPeriod; ++t) {
    for (int s = t - M + 1; s <= t; ++s) {
        for (int k = 0; k < K; ++k) {
            double sigma = getStandardDeviation(t - s, k);
            double mu = - sigma * sigma / 2;
            double r = random.nextGaussian(mu, sigma);
            Key key = new Key(s, t, k);
            map.put(key, r);
        }
    }
}
\end{lstlisting}

The multiplicative MMFE demand model
supports the notion of forecast levels
in the \code{getDemandForecast()} method.
\begin{itemize}
\item
For all $t \leq u$,
the ``myopic'' forecast at period $t$ for the demand during period $u$ is
\[
    \rvD_{tu}^{\text{myo}}
  =
    \bar{D}_u \prod_{i=u-M+1}^u \exp(\rvnu_{iu}),
\]
so that the mean and variance of the forecast do not depend on $t$.
For the ``myopic'' forecast,
the variance of the forecast is not reduced over time
as the demand forecast is updated with more current demand information
that is revealed closer to period $u$.
The ``myopic'' forecast is implemented in the code
by setting the parameter \code{forecastLevel = 0}.

\item
For forecasts far ahead in the future, $t < u - M + 1$,
the ``industry'' forecast at period $t$ for the demand during period $u$ is
\[
    \rvD_{tu}^{\text{ind}}
  =
    \bar{D}_u \prod_{i=u-M+1}^u \exp(\rvnu_{iu}).
\]
For forecasts closer in the future, $u - M + 1 \leq t \leq u$,
the ``industry'' forecast at period $t$ for the demand during period $u$ is
\[
    \rvD_{tu}^{\text{ind}}
  =
    \bar{D}_u
    \prod_{i=u-M+1}^{t-1} \exp(\nu_{iu})
    \prod_{i=t}^u \exp(\rvnu_{iu}),
\]
i.e.\ for $i = u-M+1,\ldots,t-1$,
the realizations of the random variables $\rvnu_{iu}$
have been revealed as $\nu_{iu}$,
so the forecast mean is updated to
$\bar{D}_u \prod_{i=u-M+1}^{t-1} \exp(\nu_{iu})$
and the forecast variance is reduced.

\end{itemize}

The implementation of the \code{getDemand()} method is straightforward.
Compute $\rvD_u$
by computing the sum of the values of the $\rvnu_{iu}$ variables
and doing arithmetic.
\begin{lstlisting}
int getDemand(int t) {
    double sum = 0;
    for (int s = t - M + 1; s <= t; ++s) {
        for (int k = 0; k < K; ++k) {
            Key key = new Key(s, t, k);
            sum += map.get(key);
        }
    }
    return (int) Math.round(getDemandMean(t) * Math.exp(sum));
}
\end{lstlisting}





\subsection{Vehicle Accessiblity Lead Time Model}

The vehicle accessibility lead time model is implemented
in the class \code{Vehicle\-Accessibility\-Single\-Facility\-Lead\-Time}.
This lead time model assumes that
when a district facility makes a shipment to a health facility
(in the intermediate stocking configuration)
or receives a shipment from the national facility for a health facility
(in the cross-docking configuration),
the district facility will deliver the shipment
in its next visit to the health facility.
The model assumes that
there is a Bernoulli random variable representing the vehicle availability,
and a second Bernoulli random variable representing
the roads to the health facility being accessible by vehicles.
A successful visit to a health facility
occurs during a given period
if and only if both of the Bernoulli random variables are successes.
For each health facility,
the vehicle availibility is given by the mean secondary lead time
for health facilities in the district (see \code{replenishment.csv});
while the health facility accessibility
is specified in the input file \code{health-facilities-accessibility.csv}.

To instantiate the class \code{VehicleAccessibilityLeadTime},
we must specify the following parameters:
\begin{itemize}
\item The simulation horizon,
i.e.\ generate visit booleans (true/false)
between \code{startPeriod} and \code{endPeriod}
\item \code{mean} The mean lead time if the destination facility
had 100\% accessibility
\item \code{accessibility}
A \code{double} array,
with \code{accessibility[k]} representing the probability that
the destination facility is accessible in periods $t \equiv k \amodb 48$
\item \code{random}
An Apache Commons Math \code{RandomDataGenerator} object
to generate random variables
\end{itemize}

The constructor of the class \code{VehicleAccessibilityLeadTime}
creates a Bernoulli random variable for each period $t$
between the \code{startPeriod} and \code{endPeriod},
which indicates whether the vehicle
makes a visit to the retail facility during period $t$.
The probability that the Bernoulli random variable for period $t$ is a success
is the product of the accessibility of the retail facility
and the vehicle availibility probability.
In the constructor,
the following code sets hasVisit(t)
equal to the boolean (true/false)
depending on whether there is a successful visit to the health facility
in the period $t$.
\begin{lstlisting}
hasVisit = new NegativeArray<>(startPeriod, endPeriod);
for (int t = startPeriod; t < endPeriod; ++t) {
    double r = random.nextUniform(0.0, 1.0);
    double vp = visitProbability(t);
    hasVisit.set(t, r < vp);
}
\end{lstlisting}
To compute the next lead time for a shipment sent in period $t$,
the code just needs to look for the next period $\geq t$
which is a successful visit.
If there is no visit within the simulation horizon,
then indicate that the lead time does not exist.
\begin{lstlisting}
int getLeadTime(int period) throws NoSuchElementException {
    for (int t = period; t < endPeriod; ++t) {
        if (hasVisit.get(t)) {
            return t - period;
        }
    }
    String errorMessage = String.format(
            "startPeriod = %d, endPeriod = %d, period = %d",
            startPeriod, endPeriod, period);
    throw new NoSuchElementException(errorMessage);
}
\end{lstlisting}