\documentclass{beamer}

\usepackage{booktabs}

\usetheme{Warsaw}

\newcommand{\code}[1]{\texttt{#1}}

\title{High-Level Structure of Supply Chain Simulator Manual}
\author{Ngai-Hang Zachary Leung}

\begin{document}

\begin{frame}
\maketitle
\end{frame}


\begin{frame}
\tableofcontents
\end{frame}

\section{Conceptual Model}

\begin{frame}{Goal of this Section}
\begin{itemize}
\item
High-level description of the implementation of the logical structure in code
\item
How to implement your own demand model, lead time model or replenishment policy
\end{itemize}
\end{frame}


\begin{frame}{Overview}
\begin{itemize}
\item SCS simulates a country's supply chain for a single product
\item SCS is a discrete-event simulation
\item Refer to sections below for supply chain topology,
exogenous supply and demand,
inventory replenishment process
\end{itemize}
\end{frame}

\begin{frame}{Supply Chain Topology}
\begin{itemize}
\item The supply chain topology can be represented by a graph
\item The nodes correspond to the facilities
\item The directed edges correspond to a (supplier, customer) relationship
\item The SCS supply chain has three tiers
\item Each retail facility is located in a geographical region
\end{itemize}
\end{frame}

\begin{frame}{Exogenous Supply and Demand}
\begin{itemize}
\item The national supply schedule specifies
the timing and quantity of inventory arriving at the national facility
\item At each retail facility,
inventory is deplemented by demand induced by customers
\item The demand is random and generated from a demand model
\item The demand model is also able to generate demand forecasts
\end{itemize}
\end{frame}





\subsection{Inventory Replenishment Process}

\begin{frame}{Overview}
\begin{itemize}
\item We define the inventory replenishment process
  and the sequence of events
  in the context of a generic (supplier, customer) relationship
\item We define primary and secondary distribution,
  and the associated shipment schedule, shipment lead times,
  and report lead times
\item We define the inventory replenishment process
  in an intermediate stocking configuration
\item We define the inventory replenishment process
  in a cross-docking configuration
\end{itemize}
\end{frame}



\begin{frame}{Generic (Supplier, Customer) Relationship}
\begin{itemize}
\item We define the inventory replenishment process
  in the context of a generic (supplier, customer) relationship
\item We first define the following entities:
  report, replenishment policy and shipment
\item The inventory replenishment process is:
    \begin{itemize}
    \item
    The customer facility sends a report
    \item
    The supplier facility receives the report,
    calculates a shipment quantity
    according to the inventory replenishment policy,
    and makes a shipment to the customer facility
    \item
    The customer facility receives the shipment
    \end{itemize}
\item The schedule of events can be defined by
the shipment schedule indicates which periods
the supplier facility is eligible to make a shipment to the customer facility
\end{itemize}
\end{frame}

\begin{frame}{Generic (Supplier, Customer) Relationship}
\begin{itemize}
\item Next we define the schedule of events in the
  inventory replenishment process
\item We first define the following concepts:
  shipment schedule, shipment lead time, report lead time
\item The schedule of events:
    \begin{description}
    \item[Period $t - r$:]
    The customer facility sends a report
    \item[Period $t$:]
    The supplier facility receives the report
    and makes a shipment to the customer facility
    \item[Period $t + \ell_t$:]
    The customer facility receives the shipment
    \end{description}
\end{itemize}
\end{frame}




\begin{frame}{Intermediate Stocking Configuration}
\begin{itemize}
\item Regional facilities hold a central pool of inventory and make shipments
\item The national facility supplies the regional facilities.
The report lead time is the primary report lead time,
the shipment lead time is the primary shipment lead time.
\item A retail facility is supplied by its regional facility.
The report lead time is the secondary report lead time,
the shipment lead time is the secondary shipment lead time.
\end{itemize}
\end{frame}

\begin{frame}{Cross-docking Configuration}
\begin{itemize}
\item Regional facilities do not hold a central pool of inventory,
instead reports and shipments pass through
\item The national facility supplies the retail facilities.
The report lead time is
the sum of the primary and secondary report lead times,
the shipment lead time is
the sum of the primary and secondary shipment lead time.
\item The shipment schedule is the primary shipment schedule.
We assume that the regional facility will forward a shipment
to the retail facility once the regional facility receives it
\end{itemize}
\end{frame}




\begin{frame}{Comparing Inventory Replenishment Policies}
\begin{itemize}
\item What is assumed to be fixed (exogenous):
  \begin{itemize}
  \item National supply schedule
  \item Demand model
  \item Primary and secondary shipment schedule
  \item Primary and secondary report lead times
  \item Primary and secondary shipment lead time model
  \end{itemize}
\item What can be changed:
  \begin{itemize}
  \item The supply chain configuration
    (either intermediate stocking or cross-docking)
  \item The shipment quantities
  \end{itemize}
\end{itemize}
\end{frame}



\section{Computer Model}

\begin{frame}{Goal of this Section}
\begin{itemize}
  \item Explain the high-level design of the code
  \item Show the correspondence between logical entities
    and Java classes in the code
  \item Explain how a user can write his/her own
    replenishment policy, lead time model, or demand model
\end{itemize}
\end{frame}

\begin{frame}{Entities}
\begin{itemize}
\item
Each entity described in the logical structure of the supply chain
is represented by a Java class.
\item
The table below shows the correspondence
between logical entities and Java classes:
{ \footnotesize \begin{tabular}{ll}
\toprule
  Entity & Class \\
\midrule
  Supply chain topology
    & \code{Topology} \\
  Replenishment policy for national facility
    & \code{NationalReplenishmentPolicy} \\
  \ldots & \ldots \\
\bottomrule
\end{tabular}}
\end{itemize}
\end{frame}

\begin{frame}{Packages}
\begin{itemize}
\item
The code is organized into packages
\item
Here are the packages
\item
These are the contents of package A
\item
These are the contents of package B
\end{itemize}
\end{frame}

\end{document}


\begin{frame}{}
\begin{itemize}
\item
\end{itemize}
\end{frame}
